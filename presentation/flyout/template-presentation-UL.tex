% Created 2017-12-14 jeu. 14:09
% Intended LaTeX compiler: pdflatex
\documentclass[10pt,svgnames,fragile]{beamer}
               \usepackage[english]{babel}
\usepackage[utf8x]{inputenc}
\usepackage{lmodern}
\usepackage[T1]{fontenc}
\usepackage{graphicx}
\usepackage{etex}
\usepackage{xcolor}
\usepackage[normalem]{ulem}
\usepackage{textcomp}
\usepackage{pdflscape}
\usepackage{marvosym}
\usepackage{wasysym}
\usepackage{amssymb}
\usepackage{amsmath}
\usepackage{amsthm}
\usepackage{qtree}
\usepackage{bussproofs}
\usepackage{proof}
\usepackage{fitch}
\usepackage{cancel}
\usepackage{url}
\usepackage{smfthm,adjustbox}
\AtBeginSection[]{\begin{frame}<beamer>\frametitle{}\tableofcontents[currentsection,hideothersubsections]\end{frame}}
\subtitle{}

%\titlegraphic{\includegraphics[height=1.5cm]{udl}}
\institute[University of Houston]{Department of Economics \\ University of Houston  }	       \usetheme{CambridgeUS}
\usepackage{beamer_udl_theme}
\setbeamertemplate{navigation symbols}{%
	%insertslidenavigationsymbol%
	%insertframenavigationsymbol%
	%insertsubsectionnavigationsymbol%
	%insertsectionnavigationsymbol%
	%insertdocnavigationsymbol%
	%insertbackfindforwardnavigationsymbol%
}
\usetheme{default}
\author{Yeabin Moon}
\date{January, 2022}



%\title{Effects of Entry Economic Conditions\\ on the Career of Economics Ph.D.}
\title[Fly-out interview at the Lewin Group]{ \fontsize{16}{21}\selectfont Effects of Entry Economic Conditions\\ on the Career of Economics Ph.D.} 

\begin{document}



\maketitle

% * <joseph.vidal.rosset@gmail.com> 2017-12-14T13:16:35.383Z:
%
% ^.

%\section{Section 1}
%\label{sec:org0b3f57d}
%\begin{frame}[label={sec:org61b7326}]{}
%First point 
%\end{frame}
%\section{Section 2}
%\label{sec:org39e5c65}
%\begin{frame}[label={sec:org329bb0e}]{}
%\begin{block}{A title}
%Text and text again \ldots{} 
%\end{block}
%\end{frame}
%\section{Section 3}
%\label{sec:orga8cefef}
%\begin{frame}[label={sec:org19156a6}]{A bigger title}
%Text\ldots{}  \pause 
%\begin{block}{A subtitle}
%Text and bibliographical reference, for example about  Quine \cite{quine1934}
%\end{block}
%\end{frame}
%
%\begin{frame}[label={sec:orge4f2952}]{Bibliographie}
%\bibliography{../reforg}
%\bibliographystyle{smfplain}
%\end{frame}




\begin{frame}
	\frametitle{Introduction}
	\begin{quote}
		There is no unemployment among Ph.D.s in economics {\hfill-- John Siegfried}
%		There is no unemployment among Ph.D.s in economics {\small {\\\hfill-- John Siegfried}}
	\end{quote}
\vfill
	%	\epigraph{All human things are subject to decay, and when fate summons, Monarchs must obey}{\textit{Mac Flecknoe \\ John Dryden}}
	\begin{itemize}
		\item Strong demand for economics PhD over the decade (BLS 2021)
		\begin{itemize}
			\item growing demand both in academia and in practice
			\vspace{1 mm}
			\item industries appreciate causal inferences more and more  (Athey, Luca 2019)
		\end{itemize}
\vfill
		\item Pandemic left scars on the current economics profession worldwide {\footnotesize (INOMICS)}  
		\begin{itemize}
			{ \item 2020's Jobs for economists have 14\% fewer job postings than 2019}
		\end{itemize}
\vfill
		%\item {\small Bad labor market conditions at the entry have large and persistent negative effects on careers in general {\footnotesize (Kahn 2010, Oreopoulos et al. 2012)}}
		\item A depressed labor market may bear lasting scars: \textit{lost generation} 
		%\item Understanding the mechanisms leading to persistent effects of initial  conditions is fundamental in designing policies to help young workers, but less work has been done on the mechanisms driving the losses
\vfill
		\item Less work has been done on whether the careers of economists is affected by the business cycles
\vfill
		\item I build a theoretical model to examine the potential mechanisms and test the predictions empirically
\vfill		
%		how the entry condition would affect economists’ productivity and test the predictions empirically
	\end{itemize}
\end{frame}


\begin{frame}
	\frametitle{Features of the Market for Ph.D.s in Economics}
	\begin{itemize}
			\item The market for economists is ideal for studying the initial condition
	\vspace{2 mm}
			\begin{enumerate}
				\item well-defined job market
				\begin{itemize}
						\vspace{1 mm}
					\item most jobs are posted on JOE Listings
						\vspace{1 mm}
					\item recruiting process mostly occurs around ASSA conference every year 
				\end{itemize}
	\vspace{1 mm}
				\item entry-year unemployment rate is very low
				\begin{itemize}
						\vspace{1 mm}
					\item placement outcomes vary every year
				\end{itemize}
		\vspace{1 mm}
				\item different workplace environment
					\vspace{1 mm}
				\begin{itemize}
				\item academics: work under up-or-out policies
					\vspace{1 mm}
				%\vspace{.5 mm} %, in which workers who miss a set of promotion opportunities are hardly make it after
				\item private sectors:  high skilled industries
					\vspace{1 mm}
				%	\vspace{.5 mm}
				\item more than half candidates are internationals 
					\vspace{1 mm}
				\item male dominated
				\end{itemize}
			\vspace{1 mm}
				\item productivity is measurable through publishing activity
			\end{enumerate}
\vfill
			\item Detailed employment histories and a range of ranking measures are available	
\vfill
	\end{itemize}
\end{frame}



\begin{frame}
	\frametitle{Motivation and Research Question} 
	\begin{itemize}
		
		\item Workers graduating into a recession would likely match to a lower level starting jobs than their luckier counterparts (Devereux 2002)
		\begin{itemize}
			\item first job placement is important in explaining the long-term losses\\ (Kwon et al 2010, Oreopoulos et al. 2012)
			\vspace{1 mm}
			\item how long the effects remain depends on the ease of switching jobs \\(Van den Berge 2018, Cockx and Ghirelli 2016)
		\end{itemize}
		\vspace{1.5 mm}
		\item Develop the theoretical model to explain what drives the persistent outcomes for economics PhD
		\begin{itemize}
			\item academic publications are valued both in academia and practice\\ (Swindler and Goldreyer 1998)
		\end{itemize}
		\vspace{1.5 mm}
		\item Test the model's predictions using detailed information on career paths and productivity measures  available on the web
		\begin{itemize}
			\item short run: initial placements
			\vspace{1 mm}
			\item long run: occupational choices and publications 
		\end{itemize}
	\end{itemize}
\end{frame}




\begin{frame}
	\frametitle{Preview on Research Findings}
	\begin{enumerate}
		\item Demand for economists is pro-cyclical
		\begin{itemize}
			\item fluctuations are primarily driven by the academic tenure-track positions in  US
		\end{itemize}
\vfill
		\item Entering a recession is bad for placements
		\begin{itemize}
			\item  increase in unemployment rate has adverse effects on
			\begin{itemize}
				\item initial academic employment 
				\item quality of placements conditional on taking an academic job
				\item this effect declines over time
			\end{itemize}
		\item indicate an initial mismatch
		\end{itemize}
\vfill
				\item It is bad for productivity
		\begin{itemize}
%			\item less likely to work in academia in the long run
			\item recessionary cohorts publish fewer top 50 journal articles
			\item also happened to those who started their careers at top research university
		%	\item not any more likely to switch occupations
		\end{itemize}
\vfill
		\item These effects are primarily mediated through mobility			
		\begin{itemize}
			\item economists rarely switch occupations in response to  economic conditions
			\item determinant of these switching costs is development of task-specific human capital
			%				\item occupations are quite specialized
\vfill
		\end{itemize}
	\end{enumerate}
\end{frame}







\begin{frame}
	\frametitle{Road Map}
	\begin{enumerate}
		\item Literature Review
\vfill
		\item Data
\vfill
		\item Theoretical Model
\vfill
		\item Empirical Results
\vfill
		\item Conclusion
\vfill
	\end{enumerate}
\end{frame}


\section{Literature Review}
\begin{frame}
	\frametitle{Contribution I : Persistent Effects of Entry Condition} 
	\begin{itemize}
		\item Many papers analyze the effect of entry conditions on the labor market outcomes over time (Kahn 2010, Oreopoulos et al. 2012, Schwandt and von Wachter 2019, Yu et al. 2014,  Maclean, 2015, Ball 2021)
		\begin{itemize}
\vfill
			\item expand into health, marriage, divorce, fertility, crimes
			\vspace{2 mm}
			\item effects vary by education, major, race, institutional settings \\(Altonji et al. 2014, Beiler 2017, Choi et al. 2020, Liu et al.)
		\end{itemize}
\vfill
		\item Most relevant to mine is Oyer (2006), but my research is different from:
		\vfill
		\begin{enumerate}
			\item 32 Ph.D. granting programs in U.S. having more general cohort characteristics
			\vspace{2 mm}
			\item nearly complete employment histories: possible to trace non-academic careers
			\begin{itemize}
				\item examine occupational switching
			\end{itemize}
		\vspace{2 mm}
			\item the private and international demand for economists grew exponentially 
			\begin{itemize}
				\item 73 \% increase in private sector postings from 2004 -- 2012
			\end{itemize}
		\vspace{2 mm}
			\item examine the potential mechanisms underlying persistent career effects
		\end{enumerate}
\vfill
	\end{itemize}
\end{frame}









\begin{frame}
	\frametitle{Contribution II : Occupation Choice} 
	\begin{itemize}
		\item The study of exogenous shocks can help shed additional light on the determinants of career developments
\vfill
		\item Job mobility plays a crucial role in recovering from the early damages\\  (Van den Berge 2018, Cockx and Ghirelli 2016) 
\vfill
		%\item Effects vary by education levels, race, institutional settings 
		%\item Unlucky college graduates tend to work in less attractive occupations / firms
		%		\begin{itemize}
			%			\item start and stay longer in lower-wage occupations (Altonji et al. 2016) and industries (Oreopoulos et al.
			%			2012)
			%			\item higher-earning majors typically fare substantially
			%			better in recessions relative to lower earning majors
			%		\end{itemize}
		%\item I find 
		\item Human capital formation \textit{VS} Signaling 
		\begin{itemize}
			\vspace{1.5 mm}
			\item initial investment in skills specific to occupation keeps a person on a certain career trajectory (Gibbons and Waldman 2004, 2006)
			\vspace{1 mm}
			\item bad signaling from starting in a less favorable job hinders unlucky graduates to from switching occupation when recovers (Nunley et al. 2017)
		\end{itemize}
\vfill
		\item This paper provides more supporting evidence for the model of task-specific human capital 
		\begin{itemize}
			\vspace{1.5 mm}
			\item  persistent effects are driven by the very first exposure to unemployment rates
		\end{itemize}		
\vfill
	\end{itemize}
\end{frame}


\section{Data}
\begin{frame}[label=Data]
	\frametitle{Data}
		\begin{itemize}
			\item ProQuest Dissertations \& Theses Global
			\begin{itemize}
				\vspace{1 mm}
				\item collect the doctoral dissertations by institutions, year of publications, economics (related) classification, subject codes
				\vspace{1 mm}
				\item $\sim$ 4,600 graduates from top 32 programs in U.S. between 2004--2012 
			\end{itemize}
\vfill
			\item Scrape CVs on the web or Linkedin experience profile
			\begin{itemize}
				\vspace{1 mm}
				\item collect employment history until 2020
				\vspace{1 mm}
				\item demographic information
				\vspace{1 mm}
				\item could not find about 600 individual careers
			\end{itemize}
\vfill
			\item Publication information from EconLit
\vfill
			\item List of job postings from JOE
			\begin{itemize}
				\vspace{1 mm}
				\item hiring institution, position, JEL classifications, job descriptions  
			\end{itemize}		
			
\vfill
			\item Construct the matching algorithm to compile all data % \hyperlink{appendix}{\beamerbutton{appendix}}
\vfill
		\end{itemize}
	\end{frame}

\begin{frame}
	\frametitle{Data Preparation Workflow}
	\includegraphics[width=\linewidth]{pic_flow.jpg}
\end{frame}



\begin{frame}
	\frametitle{Cyclical Demand for Economics PhD}
	
	\begin{figure}
		\centering
		%	\caption{\small Number of JOE Listings and U.S. institutions}
		\includegraphics[width=0.45\textwidth]{"/Users/yeabinmoon/Dropbox (UH-ECON)/JMP/writing/figures/temp/figure1".png} 
		\includegraphics[width=0.45\textwidth]{"/Users/yeabinmoon/Dropbox (UH-ECON)/JMP/writing/figures/temp/figure3_update".png} 
		%	\captionsetup{width=1.0\textwidth}
		%	\caption*{\footnotesize For the left panel, I count the number of job postings in JOE Listings by sections. The numbers are totalled based on academic year not calender year. For the right panel, I summarzie the count in the Carnegie Classification of Institutions of Higher Education in 2010.}
	\end{figure}
	

	\begin{itemize}	
		\item Total postings decreased by 22 percent between 2008--2010
		\vspace{2 mm}
		\begin{itemize}
			\item largest drop: the full-time tenure track academic positions in U.S. ($\sim$ 45 \%)
		\end{itemize}
		%\item Full-time US academic postings showed the largest decline (45 percent)  
	\end{itemize}
\end{frame}


%\begin{frame}
%	\frametitle{Descriptive Statistics}
%	\input{"tables/table1"}
%\end{frame}

\section{Theoretical Model}
\begin{frame}
	\frametitle{Theoretical Framework: task-specific human capital}
	\begin{itemize}
		\item How do economists accumulate human capital?
\vfill
		\item Theory of human capital accumulation provides the foundation of career development
		\begin{itemize}
			\item human capital accumulation is largely determined during the first decade of one's career in high-skill occupations {\small (Rosen 1990, O'Flaherty and Siow 1995)} 
				\vspace{1 mm}
			\item Gibbons and Waldman (2006) suggests task-specific human capital approach to explain persistent wage effects of initial conditions 
		\end{itemize}
\vfill
		\item The idea is based on that a worker develops skills according to the tasks
		\begin{itemize}
			\item e.g. acquire teaching skills more if one placed at liberal arts college
				\vspace{1 mm}
			\item it is different to
			\begin{itemize}
				\item general-purpose human capital: education
					\vspace{1 mm}
				\item firm (or occupation) specific human capital
			\end{itemize}
		\end{itemize}
\vfill
		\item Job mobility would raise  questions on the transferability of skills 
		\begin{itemize}
			\vspace{1 mm}
			\item more costly for  whose skills are not transferable across jobs
			\vspace{1 mm}
			\item critical at research universities in which early switching tends to be discouraged 						
		\end{itemize}
\vfill
	\end{itemize}
\end{frame}



\begin{frame}[label = job_des]
	\frametitle{Division of Tasks by Occupation}
	\begin{itemize}
		\item Literature use occupational and industry codes from the census
		\begin{itemize}
			\vspace{1.5 mm}
			\item change in occupation means the skills required for new occupations would be substantially different from those used in the old 
			\vspace{1.5 mm}
			\item need to build another index because of the small range of occupations economists would work at
		\end{itemize}
\vfill
		\item Define occupations  by analyzing job descriptions and other sources
		\begin{itemize}
		\vspace{1.5 mm}
			\item R1 university defined by {\small Carnegie Classifications}
			\vspace{1.5 mm}
			\item All other universities in US
			\vspace{1.5 mm}
			\item Research organization or governmental agencies in US (e.g. World Bank)
			\vspace{1.5 mm}
			\item Foreign institute
			\vspace{1.5 mm}
			\item Private institute\\
			\vspace{1.5 mm}
			\hyperlink{NLP}{\beamerbutton{appendix}}
		\end{itemize}
\vfill
	\end{itemize}
\end{frame}



\begin{frame}
	\frametitle{Model Preview}
	\begin{itemize}
		\item Based on the model from  Gibbons and Waldman {\small (2004 and 2006)}
\vfill
		\begin{itemize}
			\item simplify the accumulation speed and the effect of schooling
\vfill
			\item follow the definition of occupation
			\begin{itemize}
				\vfill
				\item  define occupation $o$ as the collection of firms having the same task
				\vfill
				\item add the explicit task weights on each occupation $o$
			\end{itemize}
\vfill
\item add the dynamics of task-specific human capital jointly determined by innately ability and labor market experience 
\vfill
		\item the output is determined by match qualities of human capital, ability, and firm with a worker
		\vfill
	\item incorporate entry economic conditions into the model
		\begin{itemize}
			\vfill
			\item predict the worker's mobility
		\end{itemize}
		\vfill
		\end{itemize}
\end{itemize}
\end{frame}



\begin{frame}
	\frametitle{Model}
		\begin{itemize}
	\item Worker $i$ at firm $f$ in $o$ at $t$ produces cumulative task $j$ specific output $Y_{ifot}^j$:
	\begin{align}
		\log{Y_{ifot}^j} = \gamma_o&\overbrace{\left[  \sum_j \beta_{o}^j \left( \underbrace{H_{it}^{j}}_{\beta_{o'}^j \text{Exp}_{io't}}\right)  \right]}^{\text{Task}_{iot}} + {\overbrace{\sum_j \beta_{o}^j \alpha_{i}^j }^{\text{m}_{io}}+\mu_{if}}\\
		&\text{where} \sum_{j} \beta_{o}^j = 1 \;\;\text{for all }o=1,...,O \notag
	\end{align}

		\item $\gamma_o$ is the occupation-specific return to human capital
		\item $\beta_{o}^{j} $ is the share of time a worker spends on average in the task $j$ in  $o$
		\item $H_{it}^j$ is the human capital accumulated in task $j$ until time period $t$
		\begin{itemize}
			\item $\text{Exp}_{io't}$ denotes the previous tenure in occupation $o'$ to simplify exposition
		\end{itemize}
		\item  $\alpha_{i}^j$: initial endowment for the task $j$
		\item $\mu_{if}$ denotes the idiosyncratic match quality between $i$ and $f$ 
	
		\end{itemize}
	
\end{frame}


\begin{frame}
\frametitle{Model - continue}
\begin{align}
	\log{Y_{ifot}^j} = \gamma_o{\text{Task}_{iot}} +{\overbrace{\sum_j \beta_{o}^j \alpha_{i}^j }^{\text{m}_{io}}}+\mu_{if} \;\;\text{where} \sum_{j} \beta_{o}^j = 1 \;\;\text{for all }o=1,...,O 
\end{align}
\begin{itemize}
%	\item assume that match quality with a firm $\mu_{if}$ random
	\vspace{3 mm}
	\item Hence, the output is determined by the match qualities
	\begin{enumerate}
		\vspace{1 mm}
		\item between the tasks and the accumulated human capital
		\begin{itemize}
			\item task tenure
		\end{itemize}
		\vspace{1 mm}
		\item  between the tasks and  the innate ability
		\vspace{1 mm}
		\item between the firm $f$ and individual $i$
	\end{enumerate}
\end{itemize}
\end{frame}

\begin{frame}
	\frametitle{Characteristics of task tenure}
	\begin{itemize}
		\item To make an exposition simpler, examine two-task model $J = \left\lbrace R, T \right\rbrace $
		\begin{equation*}
			\text{Task}_{iot}=\gamma_o\left\lbrace  \beta^R_o H_{it}^R + \left(1-\beta^R_o\right) H_{it}^T  \right\rbrace 
		\end{equation*}
\vfill
	\item $o'$ and $o$ denote source and target occupation, respectively
\vfill
\begin{block}{Proposition}
	For $\beta_{o'}^R>0.5$, task-tenure is valued more if moves to $\beta_o^R > \beta_{o'}^R$\\
	For $\beta_{o'}^R<0.5$, task-tenure is valued more if moves to $\beta_o^R < \beta_{o'}^R$\\
	For $\beta_{o'}^R=0.5$, task-tenure does not change regardless of moving 
\end{block}
\vfill
	\item If $o$ is more specialized than $o'$, the one's task tenure would be valued more
	\begin{itemize}
		\item e.g. if one worked at teaching college, her task-tenure would be valued less when moving to research-heavy university
	\end{itemize}
\vfill
	\item if $o'$ is very general, switching does not have any merit for task tenure
\vfill
	\end{itemize}
\end{frame}




%
%\begin{frame}
%	\frametitle{Occupational Choice}
%	\begin{itemize}
%		\item Assuming that economics PhDs are research-oriented, pro cyclical demand for economists results in  mismatch during the bad times at entry
%		\begin{itemize}
%				\vspace{2 mm}
%			\item $m_{io}$ reflects economic conditions
%				\vspace{2 mm}
%			\item initial productivity gap is mostly driven by $m_{io}$ since $H_{it}^j=0$ at entry
%		\end{itemize}
%	\vspace{3.5 mm}
%		\item If one did not switch, the task tenure is evolved according to the tasks
%	\vspace{3.5 mm}
%		\item Now consider one's occupational choice
%		\begin{itemize}
%				\vspace{2 mm}
%			\item bad economy placed $i$ at a teaching college
%				\vspace{2 mm}
%			\item after a few years, $i$ has options for working at research heavy occupation $o$
%				\vspace{2 mm}
%			\item switching would raise $m_{io}$ but risk losing task tenure
%		\end{itemize}
%		\vspace{2 mm}
%	\end{itemize}
%\end{frame}
%


	
	
	\begin{frame}
		\frametitle{Occupational switching}
		\begin{itemize}
		\item Improvement on match-up qualities and returns to task tenure would make a shift more likely, but there is a loss from the task tenure according to the proposition when move
	\begin{align}	\label{eq:eq10}
		\left(m_{io}-m_{io'} \right) + &\left( \mu_{if}-\mu_{if'}\right)+\left( \gamma_o-\gamma_{o'}\right) \text{Task}_{io't} \\
		&>\gamma_o\underbrace{\left[ \left( \beta_{o'}^R - \beta_{o}^R \right) \left( H_{it}^R-H_{it}^T\right) \right]}_{\text{potential loss}} +  \underbrace{x_{o't}}_{\text{switching cost}} \notag
	\end{align}
		\vspace{3 mm}
		\item Potential loss is governed by two factors
		\vspace{1 mm}
		\begin{itemize}
			\item how similar the tasks between occupation $o$ and $o'$, $|\beta_o^R-\beta_{o'}^R|$
			\begin{itemize}
				\item if the source occupation is very general, there would be no loss
			\end{itemize}
			\vspace{1 mm}
			\item how much human capital accumulated from the previous occupations
		\end{itemize}
		\end{itemize}
\end{frame}





\begin{frame}
	\frametitle{Discussion: Overview of the model's contributions}
	\begin{itemize}		
		\item  If economists’ human capital is not task-specific, the markets would be similar to the high skilled industry
		\vspace{2 mm}
		\begin{itemize}
			\item  the workers would mitigate the initial mismatch by switching, and hence the effects would not be permanent 
		\end{itemize}
\vfill
		\item If workers' human capital is task specific, there are two more cases
		\vspace{2 mm}
		\begin{itemize}
			\item the economist’s tasks are specialized (distances  are significant)
			\begin{itemize}
				\item they would less likely switch because they might risk losing the human capital 
				\vspace{2 mm}
				\item the initial effects would remain 
			\end{itemize}
			\item the economist’s tasks are general (distances are small)
			\vspace{2 mm}
			\begin{itemize}
				\item  economists would more easily switch the occupation, and hence the initial placement effects are less likely to be permanent
			\end{itemize}
		\end{itemize}
		
	\end{itemize}
\end{frame}



\section{Empirical Results}
\begin{frame}
	\frametitle{Empirical Strategy I}
	\begin{itemize}
		\item The regression model is not typically directly derived from the  model
		\vfill
		\item Estimate the short- and long-term effects of initial labor market conditions 
			\begin{itemize}
				\item for individual $i$, cohort $c$, department $d$, fields of study $f$
				\begin{equation}
					y_{icdf} = \beta \text{ec}_{c}+\gamma X_{i}+\lambda_{d}+\theta_f  +\epsilon_{icdf}
				\end{equation}
				where $ec_c$ indicates the economic conditions at graduation for $c$
				\item $\lambda_{d},\theta_f $ are fixed effects for department and fields of study, respectively
				\item $X_i$ includes US bachelor and gender indicators
	\end{itemize}
	\vfill
			\item Investigate $y_{icdf}$:
			\begin{enumerate}
				\item Placement outcomes
				\begin{itemize}
					\item whether one would be landed at R1 university as a full-time professor
					\item ranking of the placements
				\end{itemize}
				\item Job mobility
				\begin{itemize}
				\item whether an individual has ever switched from the initial placements
				\end{itemize}
	\vfill
			\end{enumerate}
\end{itemize}
\end{frame}


\begin{frame}
	\frametitle{Empirical Strategy II}
	\begin{itemize}
		\item Analyze the effect of entry conditions on the economists' productivity
			\vfill
		\item Approximate the productivity using the cumulative number of publications
			\vfill
		\begin{itemize}
			\item for individual $i$, cohort $c$, department $d$, fields of study $f$, at year $t$
			\begin{equation}
				y_{icdft} = \beta \text{ec}_{c}+\gamma X_{i}+\lambda_{d}+\theta_f +\tau_{exp}  +\epsilon_{icdft}
			\end{equation}
			where $ec_c$ indicates the economic conditions at graduation for $c$
			\item $\lambda_{d},\theta_f $ are fixed effects for department and fields of study, respectively
			\item $\tau_{exp}$ capture the labor-market experience fixed effects 
			\item $X_i$ includes US bachelor and gender indicators
		\end{itemize}
			\vfill
		\item Investigate $y_{icdft}$:
		\begin{itemize}
			\item Cumulative number of articles
			\begin{itemize}
				\item top 50, top 20 and top 5
			\end{itemize}
				\vfill
		\end{itemize}
	\end{itemize}
\end{frame}




%
%
%\begin{frame}
%	\frametitle{Prediction 2: Long-run Placements}
%	\begin{itemize}
%		\item I further test whether the entry economic conditions predict the occupational choice in the long run
%		\begin{itemize}
%			\item Using the same specification (\ref{eq:eq1}), the dependent variable is whether one work at R1 university five and nine years after graduation
%		\end{itemize}
%		\vspace{2 mm}
%		\item The model predicts that the effect will remain if economists develop task-specific human capitals 
%		%\item Find that unemployment has a significant impact on the initial placements as expected
%		\vspace{2 mm}
%		\item Also, if one had a higher switching cost, the effects would be stronger
%		%		\begin{itemize}
%			%			\item also, if one had a higher switching cost, the effects would be stronger
%			%		\end{itemize}
%	\end{itemize}
%\end{frame}
%
%
%
%
%\begin{frame}
%	\frametitle{Prediction 3: Productivity}
%	\begin{itemize}
%		\item Now I test whether the entry economic conditions would affect the economists' productivity
%		\begin{itemize}
%			\item main measures of research output for academic economists are their publications
%			\item for individual $i$, cohort $c$,  department $d$, field of study $f$, year $t$, labor market experience $exp$
%			\begin{equation*}
%				y_{icdft} = \beta \text{ec}_{c}+\gamma X_{i}+\xi_{d}+\theta_f +\mu_{exp}  +\epsilon_{icdft}
%			\end{equation*}
%			where $ec_c$ indicates the economic conditions at graduation for $c$
%			%\item tenure in the profession, publication outcomes, time to tenure, mobility
%			\item $y_{icdft}$ is the number of publications in top  economics journals
%			\item $X_i$ includes an indicator for receiving bachelor degrees in the U.S. and gender 
%		\end{itemize}
%		%\item $exp$ denotes years of potential labor market experience
%		%		\item $\text{exp}_{it}$ denotes the potential labor market experience
%		%		\begin{itemize}
%			%			\item only makes sense when outcome is not specific to certain occupations like wage
%			%			\item if $Y$ is num publications, it should count  academic experience only (updating)
%			%		\end{itemize}	
%		%		\item Identifying assumption in Model I would be weak
%		%		\begin{itemize}
%			%			\item those who end up in better jobs later may
%			%		\end{itemize}
%	\end{itemize}
%\end{frame}




{
	\setbeamerfont{frametitle}{size=\small}
\begin{frame}
	\frametitle{Table 2: Effect of entry condition on the initial placements in R1}
	\input{"tables/table2"}
\end{frame}}

{
	\setbeamerfont{frametitle}{size=\small}
\begin{frame}
	\frametitle{Table 3: Effect of entry condition on  initial-placement ranking   for those who placed in R1}
	\input{"tables/table3"}
\end{frame}}


\begin{frame}
	\frametitle{Discussion}
	\begin{itemize}
		\item Table 2 presents that the entry conditions would negatively affect the placement outcomes at R1
		\vspace{2 mm}
		\item Assuming that most graduates are research-oriented, the bad entry conditions would result in an occupational mismatch
		\vspace{2 mm}
		\item Table 3 further presents that the quality of the placement even within the R1 is also lowered by the bad economic conditions
		\vspace{2 mm}
		\item Note that faculties in more prestigious institutions tend to spend less time teaching, the terrible entry conditions would result in the task mismatch even within R1 compared to good entry conditions
		\vspace{2 mm}
	\end{itemize}
\end{frame}



{
	\setbeamerfont{frametitle}{size=\small}
\begin{frame}
	\frametitle{Table 4: Effect of entry conditions on the placement in R1  over time}
	\input{"tables/table4"}
\end{frame}}

%\begin{frame}
%	\frametitle{Table 5}
%	\input{"tables/table5"}
%\end{frame}

{
	\setbeamerfont{frametitle}{size=\small}
\begin{frame}
	\frametitle{Table 5: Effect of entry condition on  number of publications in Top 50 economics journals}
	\input{"tables/table6"}
\end{frame}}

\begin{frame}
	\frametitle{Table 6: Transition Probability between Occupations}
	\input{"tables/table7"}
\end{frame}


%\begin{frame}
%	\frametitle{Table 7: Effect of entry conditions on the Job Mobility}
%	\input{"tables/table8"}
%\end{frame}




\begin{frame}[label = robustness]
	\frametitle{Robustness Check}
	\begin{itemize}
		\item In the analysis above,  assume that the macroeconomic conditions at graduation represent an exogenous labor demand shock
		\begin{itemize}
			\vspace{1 mm}
			\item  the average quality of graduates who enters the market is not systematically associated with the economic conditions
		\end{itemize}
\vfill
		\item 5 years of study is arguably the norm of the economics Ph.D. programs	
		\begin{itemize}
			\vspace{1 mm}
			\item start year of PhD is partially observable
		\end{itemize}
		%	\item Observe the duration of the study for 60 percent of the sample
\vfill
		\item Examine the effect of the entry economic conditions on one's decision to delay graduation % \hyperlink{delay}{\beamerbutton{appendix}}
		\begin{itemize}
			\vspace{1 mm}
			\item individuals tier 1 programs would have an option to delay 
			\vspace{1 mm}
			\item revisit the previous findings using individuals from other than tier 1 programs % \hyperlink{rank2}{\beamerbutton{appendix}}
		\end{itemize}
\vfill
	\end{itemize}
\end{frame}


\begin{frame}
	\frametitle{Table 7: Effect of entry condition on delaying graduation}
	\input{"tables/table10"}
\end{frame}

%{
%	\setbeamerfont{frametitle}{size=\small}
%\begin{frame}
%	\frametitle{Table 8: Regressions without graduates from tier 1 programs}
%	\input{"tables/table11"}
%\end{frame}}


%\begin{frame}
%	\frametitle{Discussions: Possible Concerns: REWRITE}
%	\begin{itemize}
%		\item Incomplete data extraction
%		\begin{itemize}
%			\item about 12 percent of individuals have no records from ProQuest listings 
%		\end{itemize}
%		\vspace{1 mm}
%		\item Measurement errors
%		\begin{itemize}
%			\item possible mismatch between the degree date and market entry
%			\item  some could intentionally hide the previous positions
%		\end{itemize}
%		
%		\vspace{1 mm}
%		\item Selection issues on CV / resume
%		\begin{itemize}
%			\item maybe more successful individuals would update CVs more    
%		\end{itemize}
%		\item Matching errors
%		\vspace{1 mm}
%		\begin{itemize}
%			\item Duplicated names
%		\end{itemize}
%	\end{itemize}
%\end{frame}


\section{Conclusion}
\begin{frame}
	\frametitle{Conclusion}
	\begin{itemize}
		\item The analysis cannot be complete
		\begin{itemize}
			\item it is based on the C.V. or resume, entirely subjective
			\vspace{1 mm}
			\item measurement errors
			\vspace{1 mm}
			\item attrition problem
		\end{itemize}
	\vfill
		\item Assuming that those missing individuals are less likely successful, I believe my findings would provide 	the minimum effects of the entry conditions on the economics Ph.D.'s career and productivity
		\vfill
		\item To conclude, the transition from education to the labor market in a recession would threaten the economists' careers
		\vfill
		\item Their occupational outlook would not be more 		promising than surrounding cohorts, and the productivity loss is expected on average
\vfill	
\end{itemize}
\end{frame}

\begin{frame}{}
	\centering \Huge
	\emph{Thank You}
\end{frame}

%%%%%%%%%%%%%%%%%%%%%%%%%%%%%%%%%%%%%%%%
% Appendix
%%%%%%%%%%%%%%%%%%%%%%%%%%%%%%%%%%%%%%%%

\begin{frame}[label=appendix]
	\frametitle{Fuzzy matching}
	\begin{itemize}
		\item One challenge of the task is scrape text data from the source document and convert them into suitable format
		\begin{itemize}
			\vspace{1 mm}
			\item web scraping and document scraping 
			\begin{itemize}
			\vspace{1 mm}
				\item search tools in Python or range of APIs
						\vspace{1 mm}
				\item might involve legal issues $\rightarrow$ commercial APIs
			\end{itemize}
		\end{itemize} 
	\vfill
		\item Bigger challenge is that there are same institution but were taken as different forms 
		\begin{itemize}
					\vspace{1 mm}
			\item CV, dissertations, rank data, Journal entry
					\vspace{1 mm}
			\item matching economists' names are even more complicated
		\end{itemize}
	\vfill
		\item Employ learning methods from data science literature
		\begin{itemize}
					\vspace{1 mm}
			\item data matching or fuzzy matching (probabilistic data matching)
		\end{itemize}
\vfill	
\end{itemize}
\end{frame}


\begin{frame}[label=appendix]
	\frametitle{General Steps}
	\begin{itemize}
		\item N-grams: a set of co-occurring words within a given sentence\\ (Wang et al. 2006)
		\begin{itemize}
			\item collect the words in the sentence having more meaning
		\end{itemize}
	\vfill
		\item TF-IDF: count the word occurs in each document
		\begin{itemize}
			\item evaluate how important a word is and (learning)
			\begin{itemize}
				\item very important since the names have only a few words
			\end{itemize}
			\item long computing time ...
		\end{itemize}
	\vfill
		\item Cosine similarity: how close the two sentences is
	\vfill
		\item Matching rates vary
		\begin{itemize}
			\item JOE in US institutions: 89\%
			\item All institutions: 70\%
		\end{itemize}
	\end{itemize}
	\hyperlink{Data}{\beamerbutton{back}}
\vfill
\end{frame}

%
%\begin{frame}[label = teaching]
	%	\frametitle{Hours per week teaching credit classes}
	%	\input{"tables/table2"}
	%	\hyperlink{Occupation}{\beamerbutton{slide}}
	%\end{frame}
%

\begin{frame}[label = NLP]
	\frametitle{Job description: Natural Language Processing}
	\begin{itemize}
		\item Analyze the text in the job descriptions from JOE and CSWEP letters
\vfill
		\item Most Frequently Appeared Words in job postings
		\begin{itemize}
			\vspace{1 mm}
			\item Tenured track positions: \textbf{research, economics, teaching}, curriculum
			\vspace{1 mm}
			\item Research org: \textbf{research, economics}, teaching
			\vspace{1 mm}
			\item Private: \textbf{research}, economics, communication, work, policy, experience, analysis, skills, quantitative, management
		\end{itemize}
\vfill		
		\item Word \textbf{research} and \textbf{teaching} dominates in Academic positions
\vfill
		\item Diverse range of words are captured in private sector positions
		\vspace{1 mm}
		\begin{itemize}
			\item communication related words are rarely captured in academic positions
		\end{itemize} 
\vfill
		\item Possibly, different skills are required for the private sectors \hyperlink{job_des}{\beamerbutton{back}}
\vfill
	\end{itemize}
\end{frame}





\end{document}