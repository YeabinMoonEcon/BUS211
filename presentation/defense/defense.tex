% Created 2017-12-14 jeu. 14:09
% Intended LaTeX compiler: pdflatex
\documentclass[10pt,svgnames,fragile]{beamer}
\usepackage[english]{babel}
\usepackage[utf8x]{inputenc}
\usepackage{lmodern}
\usepackage[T1]{fontenc}
\usepackage{graphicx}
\usepackage{etex}
\usepackage{xcolor}
\usepackage[normalem]{ulem}
\usepackage{textcomp}
\usepackage{pdflscape}
\usepackage{marvosym}
\usepackage{wasysym}
\usepackage{amssymb}
\usepackage{amsmath}
\usepackage{amsthm}
\usepackage{qtree}
\usepackage{bussproofs}
\usepackage{proof}
\usepackage{fitch}
\usepackage{cancel}
\usepackage{url}
\usepackage{smfthm,adjustbox}
\AtBeginSection[]{\begin{frame}<beamer>\frametitle{}\tableofcontents[currentsection,hideothersubsections]\end{frame}}
\subtitle{}

\newcommand{\REV}[1]{\textcolor{red}{#1}} 

%\titlegraphic{\includegraphics[height=1.5cm]{udl}}
\institute[University of Houston]{Department of Economics \\ University of Houston  }	       \usetheme{CambridgeUS}
\usepackage{beamer_udl_theme}
\setbeamertemplate{navigation symbols}{%
	%insertslidenavigationsymbol%
	%insertframenavigationsymbol%
	%insertsubsectionnavigationsymbol%
	%insertsectionnavigationsymbol%
	%insertdocnavigationsymbol%
	%insertbackfindforwardnavigationsymbol%
}
\usetheme{default}
\author{Yeabin Moon}
\date{June, 2022}



%\title{Effects of Entry Economic Conditions\\ on the Career of Economics Ph.D.}
\title[Dissertation defense]{ \fontsize{16}{21}\selectfont Proxying elderly cognition\\ with survey responses to financial questions} 

\begin{document}
	
	
	
	\maketitle
	
	% * <joseph.vidal.rosset@gmail.com> 2017-12-14T13:16:35.383Z:
	%
	% ^.
	
	%\section{Section 1}
	%\label{sec:org0b3f57d}
	%\begin{frame}[label={sec:org61b7326}]{}
	%First point 
	%\end{frame}
	%\section{Section 2}
	%\label{sec:org39e5c65}
	%\begin{frame}[label={sec:org329bb0e}]{}
	%\begin{block}{A title}
	%Text and text again \ldots{} 
	%\end{block}
	%\end{frame}
	%\section{Section 3}
	%\label{sec:orga8cefef}
	%\begin{frame}[label={sec:org19156a6}]{A bigger title}
	%Text\ldots{}  \pause 
	%\begin{block}{A subtitle}
	%Text and bibliographical reference, for example about  Quine \cite{quine1934}
	%\end{block}
	%\end{frame}
	%
	%\begin{frame}[label={sec:orge4f2952}]{Bibliographie}
	%\bibliography{../reforg}
	%\bibliographystyle{smfplain}
	%\end{frame}
	
	
	
	
	\begin{frame}
		\frametitle{Introduction}
			\vfill
		\begin{itemize}
			\item Cognitive decline is one of the most common conditions among the elderly
			\begin{itemize}
				\item 10.8 \% among 45--64 years old and 11.7 \% among 65 years and older (CDC)
			\end{itemize}
			\vfill
			\item Interest in maintaining a healthy later life has increased, focusing on the cognitive function being central to this interest
			\vfill
	
			%\item A decline in cognitive ability might be expected as people age, but stressful life events or medical incidents could increase the risk of cognitive decline

			\vfill
			\item A growing amount of evidence suggests further investigation of the cause and effect of cognitive decline
			\vfill
			\item  However, the lack of data availability limits the scope of research
			\vfill
			\item In this paper, I propose a measure as a proxy of cognitive ability based on the response patterns of popular surveys
			\vfill		
		\end{itemize}
	\end{frame}


\begin{frame}
	\frametitle{Features of Survey Response}
	\begin{itemize}
		\item Participating in a survey requires mental effort %(Cannell, Miller, and Oksenberg 1981)
			\vfill		
	   \item Incomplete answers are frequently observed in open-ended questions
	   \begin{itemize}
	   		\vfill		
	   	\item round numbers or missing responses
	   \end{itemize}
			\vfill		
		\item Focus on the open-ended \textit{financial} questions
		\begin{itemize}
				\vfill		
			\item available in popular survey-type microdata
				\vfill		
			\item the financial questions demands advanced mental operations \\(Riddles et al. 2017)
				\vfill		
			\item financial literacy is often associated with cognitive abilities \\(Cole and Shastry 2008, Bucher-Koenen and Ziegelmeyer 2011)
		\end{itemize}
		\vfill		
		\item Analyzing the response patterns of the financial quetsions would provide a better understanding of cognitive functioning
	\end{itemize}
\end{frame}

\begin{frame}[label=Motivation]
	\frametitle{\REV{Motivation}}
	\begin{itemize}
		\item Stylized patterns are observed in open-ended questions
		\begin{itemize}
			\item strong patters of heaping on  approximate number, rounded to  precision of 1\\ e.g. 2,000 or 10,000 \hyperlink{figure1}{\beamerbutton{figure}}
			\item opt-out trends in aging 
		\end{itemize}
		\item  Assume that the response format is related to cognitive ability
		\item Construct the proxy based on the response formats on the selected questions in the HRS
		
		\item Apply a similar method to construct the proxy using the PSID data
	\end{itemize}
\end{frame}

\begin{frame}
	\frametitle{Preview on Findings}
	\begin{itemize}
		\item The proxy starts to decline from the mid-60s
		\item It is positively correlated with correlated with cognitive ability, directly measured by the HRS
		\item Similar patterns are observed in the PSID
		\item As an application of the project, the impact of the Title VI of the 1964 Civil Rights Act on black people's cognitive development
		\begin{itemize}
			\item find some evidence to support this claim
		\end{itemize}
	\end{itemize}
\end{frame}


\begin{frame}
	\frametitle{Road Map}
	\begin{enumerate}
		\item Literature Review
		\vfill
		\item Data
		\vfill
		\item Framework 
		\vfill
		\item Validation
		\vfill
		\item PSID Application
		\vfill
		\item Conclusion
		\vfill
	\end{enumerate}
\end{frame}


{
	\AtBeginSection{}
	\section{Literature Review}
	\begin{frame}
		\frametitle{Literature Review} 
		\begin{itemize}
			\item Opt-out answers are considered as taking cognitive shortcuts to make answering easier (Colsher and Wallace 1989, Krosnick 1991, Kn{\"a}uper et al. 1997)
			\begin{itemize}
				\vfill
				\item old respondents are more likely to skip or choose opt-outs
				\item the respondents with low cognitive ability tend to opt out of the question more on difficult questions
				%\vspace{2 mm}
							\end{itemize}
			\vfill
			\item Another set of papers examine the relationship between cognitive ability and the pattern of numerical response (Andrews and Herzog 1986, Holbrook et al. 2014, Gideon, Helppie-McFall and Hsu 2017)
			\vfill
			\begin{itemize}
				\item strong patterns of heaping at round numbers
				\vfill
				\item rounding was more common for respondents low in ability, for respondents low in motivation, and for more difficult questions
				%			\begin{itemize}
					%				\item examine occupational switching
					%			\end{itemize}
			\end{itemize}
			\vfill
						\item To the best of my knowledge, there is no research on estimating cognitive ability using the response pattern
			
		\end{itemize}
	\end{frame}

}




{
	\AtBeginSection{}
	\section{Data}
	\begin{frame}[label = data1]
		\frametitle{Data Description} 
		\begin{itemize}
			\item Health and Retirement Study
			\begin{itemize}
				\vfill
				\item focus on the household heads aged between 50 and 89, participated in 2004--2018 survey 
				\item 62,851 individual-years, pooling 16,187 individuals across 7 waves 	\hyperlink{table1}{\beamerbutton{summary}}
				%\vspace{2 mm}
			\end{itemize}
			\vfill
			\item Question selection criteria
			\vfill
			\begin{itemize}
				\item response format should be open-ended and require more than 2-digit numbers
				\vfill
				\item response rate should be large enough
				\vfill
				\item the selected questions should appear in multiple waves 
				\vfill
				\item	10 questions are selected	
			\end{itemize}
			\vfill
		\end{itemize}
	\end{frame}
	
	
		\begin{frame}[label = data2]
		\frametitle{Response patterns in the HRS} 
	\begin{table}[ht]
	\centering
%	\caption{Descriptive statistics for the selected questions in the HRS}
	\begin{adjustbox}{max width=1\textwidth}
		%\resizebox{12 cm}{!}{%
\begin{tabular}{lllll}
	\hline\hline
	variable 	 &  	 maximal rounding (\%)&	 &Opt-out (\%)&		 \\
	 &		   	&	Don't know &	Refused to Answer &	skip\\
	\hline
Home\;value &      30.9  		& 15.6  & 1.2 & 11.9 \\
Property\;tax 	 &     26 		& 18.8  & 1.4 & 1.2\\
SSI\; income     	       &		22.6			& 12.5   & .8   & 8.3 \\
Checking 	&		48.2		 & 11    & 10.5 & .8 \\
Vehicle    			&		53.1			& 17.7  & 1.4  & .8 \\
Food\;home    			&			61		 & 9.9   & 1    & .6\\
Food\;out     			&				42		& 3.1   & .7   & .5\\
OOP\;Doc    			&		51.9		& 18.1  & .6   & 1.2\\
OOP\;Dent 			&		59.9		& 7.5   & .5   & 1.1\\
OOP\;Drug  				&		45.3	& 11.8  & .4   & 5.6\\
	\hline\hline
\end{tabular}%
\end{adjustbox}
%\caption*{\footnotesize  
%Each row summarizes the response formats. The maximal rounding indicates the numerical response format in which the number is rounded to the precision of 1. That is, the leftmost digit is any number, and the rest of the digit positions are zeros, for example, 2,000 or 100,000. The respondents in the HRS can opt out of the questions, choosing \textit{Don’t know, Refused to answer}, or skipping the question. The column \textit{skip} counts the missing responses only for those who have access to the question but skip it. The access is determined by the cross-references Core interview content (\href{https://hrs.isr.umich.edu/documentation/questionnaires}{link}). 
%The sample is limited to the household heads aged between 50 and 89 who participated in the survey from 2004 to 2018. The Social security income is available to those above 60, so the third row further restricts the sample aged between 60 and 89.
%}
%
%\label{s:tb:table2}
\end{table}
	\begin{itemize}
		\item {\small Maximal rounding indicates the numerical response, rounded to the precision of 1}
	\end{itemize}
	\end{frame}
	
			\begin{frame}
		\frametitle{Cognitive score in the HRS} 
		\begin{itemize}
			\item Following measures are selected to construct \textit{cognitive score}
			\begin{itemize}
				\item immediate recall, delayed recall, serial 7 (Herzog and Wallace,
				1995)
			\end{itemize}
		\end{itemize}
	\begin{center}
	\includegraphics[width=.6\textwidth]{"figures/figure1".png}
\end{center}
	\end{frame}
	
	
}




{
	\AtBeginSection{}
	\section{Framework}
	\begin{frame}
		\frametitle{Satisficing Theory} 
		\begin{itemize}
			\item Respondents either provide numerical values or opt out of the question
					\begin{itemize}
						\vspace{2 mm}
				\item opt-out responses indicates \textit{don't know, refused to answer, skip}
			\end{itemize}
				\vfill
			\item Providing incomplete answers are called \textit{satisficing} (Krosnick 1991)
			\begin{itemize}
				\vspace{2 mm}
				\item claim that the satisficing behavior is negatively correlated with cognitive ability
			\end{itemize}
			\vfill
		\item Investigate whether the cognitive ability could be inferred from the response patterns
			\vfill
		\item Propose a standardized method that enables cross-question comparisons in answering behavior
			\vfill
	\end{itemize}
	\end{frame}

	\begin{frame}[label=Framework2]
	\frametitle{Characterizing the response} 
	\begin{itemize}
		\item Level of rounding would help characterize the numerical response
	\begin{itemize}
		\item assume round responses is a consequence of satisficing
		\begin{itemize}
			\item Gideon et al. 2017 define the level of rounding as $\left( \frac{\text{the number of total digits}-\text{the number of significant digits}}{\text{the number of total digits}-1} \right)$  	\hyperlink{figure2}{\beamerbutton{figure}}
			\vspace{2 mm}
			\item raises the question of whether it would be sufficient to focus just on the maximal rounding \hyperlink{figure3}{\beamerbutton{figure}}

		\end{itemize}
		\end{itemize}
		\vfill
		\item The reductions in cognitive ability can be reflected in the completion rate (Kn{\"a}uper  et al. 1997)
		\begin{itemize}
			\item use a dummy indicating the opt-out responses \hyperlink{figure4}{\beamerbutton{figure}}
			
		\end{itemize}
		\vfill
		\item Ideal proxy should be responsive to the level of rounding and opt-out responses
		\vfill
		\item Divide the response format into three: opt-out, maximal rounding, and numerical answer
		\begin{itemize}
			\item assign 0,1 and 2 to each type in order
			\item assume that they represent the respondent’s cognitive ability in that order.
		\end{itemize}
		\vfill
	\end{itemize}
\end{frame}



\begin{frame}
	\frametitle{Choices of Cognitive Proxy} 
	\begin{itemize}
		\item Label the above method \textit{Moon}
		\item For comparison, employ two other ways of classifications
		\begin{itemize}
			\item \textit{Gideon}: level of rounding
			\item \textit{Kn{\"a}uper}: opt-out responses
		\end{itemize}
	\item After characterizing response using these three ways, I take the average of them
	\item To make the higher value of the proxies to imply higher cognitive ability
	\begin{itemize}
		\item use $1-level of rounding$ for Gideon
	\end{itemize}
	\item Evaluate the proxy based two aspects
			\begin{itemize}
		\item need to have trends in aging
		\item correlate with the cognitive measures
	\end{itemize}
	\end{itemize}
\end{frame}

}

{
	\AtBeginSection{}
\section{Validation}
\begin{frame}
	\frametitle{Proxy Evaluation I: Trends in Aging} 
	\begin{center}
	\includegraphics[width=.7\textwidth]{"figures/figure7".png}
\end{center}
\end{frame}

{
	\setbeamerfont{frametitle}{size=\large}
\begin{frame}
	\frametitle{Proxy Evaluation I: Trends in Aging controlling for individual effects} 
	\begin{center}
		\includegraphics[width=.7\textwidth]{"figures/figure8".png}
	\end{center}
\end{frame}}

\begin{frame}
	\frametitle{Proxy Evaluation II: Regression Analysis} 
	\begin{itemize}
		\item Cognitive development would be affected by income, education, and gender \\(Myers 1976, Budd and Guinnane 1991, Boyle and Gr{\'a}da 1986)
		\item Regression analysis on the proxies to evaluate whether the resulting associations are consistent with literature
		\item Consider: For a respondent $i$ in the survey year $t$,
	\end{itemize}
\end{frame}

}

%%%%%%%%%%%%%%%%%%%%%%%%%%%%%%%%%%%%%%%%%%%%%%%%%%%%



%%%%%%%%%%%%%%%%%% Table %%%%%%%%%%%%%%%%%%%%%%%%%%%%%%%
\begin{frame}[label = table1]
	\frametitle{Descriptive statistics for the HRS respondents} 
	% Please add the following required packages to your document preamble:
% \usepackage{graphicx}
\begin{table}[ht]
	\centering
%	\caption{Descriptive statistics of Graduates by Department Rankings}
	\begin{adjustbox}{max width=0.75\textwidth}
	%\resizebox{12 cm}{!}{%
		\begin{tabular}{lllll}
			\hline\hline
			& Overall & tier 1 & tier 2 & tier 3 \\ 
			& (1)  & (2)  & (3) & (4) \\
			\hline
			\textbf{Main independent variables} &         &        &        &        \\
			female                              & 0.287  & 0.251  & 0.324 & 0.309   \\
			& (0.452)  & (0.433) & (0.468) & (0.462) \\
			US bachelor                         & 0.426  & 0.471 & 0.398 & 0.376 \\
			& (0.494)  & (0.499) & (0.489) & (0.484) \\
			\textbf{Main outcome variables}     &         &        &        &        \\
			number of publications by 3 years &  0.327 & 0.445 & 0.247 & 0.210\\
			&  (0.742) & (0.873) & (0.627) & (0.557) \\
			number of publications by 6 years &  0.873 & 1.208& 0.641 & 0.546\\
			&   (1.516) & (1.778) & (1.298) & (1.048) \\
			number of publications by 9 years &  1.414 & 1.970 &  1.020 & 0.880 \\
			&  (2.325) & (2.748) & (1.948) & (1.562) \\
			%publications$^1$                        & 1.4156  & 1.8615 & 1.1246 & 0.9567 \\
			%& 3.0926  & 3.6232 & 2.8099 & 2.1062 \\
			%publications$^2$                        & 1.8811  & 2.5608 & 1.4264 & 1.2021 \\
			%& 3.4547  & 4.0263 & 3.0844 & 2.4057 \\
			\textbf{Initial placements}     &         &        &        &        \\
			tenure-track in R1 university                           & 0.232  & 0.301 & 0.184 & 0.165 \\
			& (0.422)  & (0.459) & (0.388) & (0.371) \\
			private Sector                      & 0.240  & 0.226 & 0.261 & 0.241 \\
			& (0.427)    & (0.418) & (0.439) & (0.427) \\ 
			
			%			\textbf{Job mobility}     &         &        &        &        \\
			%			Occupation mobility                          &.4123  & 0.4004 & 0.4244 & 0.4195 \\
			%			& (0.4923)  & (0.4901) & (0.4944) & (0.4937) \\
			%			Firm mobility                         & 0.6732 & 0.6833 & 0.6754 & 0.6519 \\
			%			& (0.4690)    & (0.4652) & (0.4683) & (0.4766) \\  
			\hline
			%			
			number of schools & 32 & 10 &10 & 12 \\
			%Observations                        & 55,582   & 23,824  & 14,735  & 17,023  \\
			number of individuals               & 3,979    & 1,795   & 1,197   & 987   \\ \hline\hline
		\end{tabular}%
	\end{adjustbox}
	%	\caption*{\footnotesize I collect the job market candidates from 32 universities in the US and group them into three categories by department rankings. The ranks are quoted from \textit{econphd.net rankings 2004}. Column (2), (3) and (4) summarize those who graduated from 1--10, 11--23, 24--45 departments in the US, respectively. I count the number of cumulative publications from the top 50 economics journals for the primary analysis. Job mobility reports the probability of switching a job from the initial placement. Standard errors are in parentheses.}
\label{tb:table1}
\end{table}
	\hyperlink{data1}{\beamerbutton{back}}
\end{frame}


%%%%%%%%%%%%%%%%%% Figure %%%%%%%%%%%%%%%%%%%%%%%%%%%%%
\begin{frame}[label = figure1]
	\frametitle{Descriptive statistics for the HRS respondents} 
	\begin{center}
	\includegraphics[width=.7\textwidth]{"figures/figure2".png}
	\end{center}
	\hyperlink{Motivation}{\beamerbutton{back}}
\end{frame}

\begin{frame}[label = figure2]
	\frametitle{Level of rounding by age} 
	\begin{center}
		\includegraphics[width=.7\textwidth]{"figures/figure3".png}
	\end{center}
	\hyperlink{Framework2}{\beamerbutton{back}}
\end{frame}

\begin{frame}[label = figure3]
	\frametitle{Maximal rounding by age} 
	\begin{center}
		\includegraphics[width=.7\textwidth]{"figures/figure4".png}
	\end{center}
	\hyperlink{Framework2}{\beamerbutton{back}}
\end{frame}

\begin{frame}[label = figure4]
	\frametitle{Opt-out trend by age} 
	\begin{center}
		\includegraphics[width=.7\textwidth]{"figures/figure5".png}
	\end{center}
	\hyperlink{Framework2}{\beamerbutton{back}}
\end{frame}

\end{document}