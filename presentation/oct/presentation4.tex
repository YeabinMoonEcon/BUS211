\documentclass[11pt]{beamer}
\usepackage[utf8]{inputenc}
\usepackage[T1]{fontenc}
\usepackage{lmodern}
\usetheme{Warsaw}
\usefonttheme{serif}
\usepackage{pgf-pie}  
\usepackage{epigraph}

\usetheme{AnnArbor}
\begin{document}
	\author{Yeabin Moon}
	\title{Effects of Entry Economic Conditions\\ on the Career of Economics Ph.D.}
	%\subtitle{}
	%\logo{}
	\institute{University of Houston}
	\date{October, 2021}
	%\subject{}
	%\setbeamercovered{transparent}
	\setbeamertemplate{navigation symbols}{}
\setbeamertemplate{page number in head/foot}{\insertframenumber\,/\,\inserttotalframenumber}
	
	%	\begin{frame}[plain]
			\maketitle
	%	\end{frame}
	
	
	\begin{frame}
		\frametitle{Introduction}
				\begin{quote}
				There is no unemployment among Ph.D.s in economics {\small {\\\hfill-- John Siegfried}}
				\end{quote}
				%	\epigraph{All human things are subject to decay, and when fate summons, Monarchs must obey}{\textit{Mac Flecknoe \\ John Dryden}}
		\begin{itemize}
			\item Strong demand for economics PhD over the decade (BLS 2021)
				\begin{itemize}
					\item growing demand both in academia and in practice
 					\item {\footnotesize industries appreciate causal inferences more and more {\scriptsize (Athey, Luca 2019)}}
				\end{itemize}
 %	\vspace{0.5mm}
			\item Pandemic left scars on the current economics profession worldwide (INOMICS) and  lowered demand for entry worker (JOE)
					\begin{itemize}
						{\footnotesize \item 2020's Jobs for economists have 14\% fewer job postings than 2019}
					\end{itemize}
%	\vspace{0.5mm}
			\item {\small Bad labor market conditions at the entry have large and persistent negative effects on careers in general {\footnotesize (Kahn 2010, Oreopoulos et al. 2012)}}
			%\item Understanding the mechanisms leading to persistent effects of initial  conditions is fundamental in designing policies to help young workers, but less work has been done on the mechanisms driving the losses
%\vspace{1mm}
			\item Less work has been done on whether the careers of economists is affected by the business cycles
			\item I build a theoretical model to examine how the entry condition would affect economists’ productivity and investigate the predictions empirically
%			\item Examine the economics Ph.D. labor market   
%			\begin{itemize}
%				    \item centralized matching systems (Coles et al. 2010) and very low unemployment rate
%					\item high skilled professionals face rigid promotion decisions early in their careers
%%					human capital accumulation and promotion tracks are very steep during the early stage of one's career 
%					%\item human capital is less general than doctors or lawyers but more general than highly educated workers
%				%	\item possible to track the career information and outcomes together with fields of study at graduation  
%			\end{itemize}
%%					\vspace{1mm}
%		 	\item The project investigates whether the initial economic conditions would affect the placement outcomes, and whether it would have left permanent effects on their careers
%		 	\begin{itemize}
%		 		\item research on beyond top 10 universities are especially thin %to my knowledge
%		 	\end{itemize}

		\end{itemize}
	\end{frame}


\begin{frame}
	\frametitle{Features of the Market for Ph.D.s in Economics}
	\begin{itemize}
		
		%			\begin{itemize}
							    \item Centralized matching systems and require advanced degrees
							    \vspace{1.5mm}
							    \item Different workplace environment
							    \begin{itemize}
							    	\item academics: work under up-or-out policies
							    	\vspace{.5 mm} %, in which workers who miss a set of promotion opportunities are hardly make it after
							    	\item private sectors:  high skilled industries
							 	    \vspace{.5 mm}
							 	    \item little is known for switching patterns among the occupations   
						     \end{itemize}
					     \vspace{1.5mm}
					     		\item More than 40 \% graduates are internationals
			   				    \vspace{1.5mm}
			   				    \item Low unemployment, but the placement outcomes varies by economic conditions
			   				   \vspace{1.5mm}
			   				    \item Detailed employment histories and some objective measures of productivity are available
			   				   % \item Cohorts began increasingly foreign and female 
			%%					human capital accumulation and promotion tracks are very steep during the early stage of one's career 
			%					%\item human capital is less general than doctors or lawyers but more general than highly educated workers
			%				%	\item possible to track the career information and outcomes together with fields of study at graduation  
			%			\end{itemize}
		%%					\vspace{1mm}
		%		 	\item The project investigates whether the initial economic conditions would affect the placement outcomes, and whether it would have left permanent effects on their careers
		%		 	\begin{itemize}
			%		 		\item research on beyond top 10 universities are especially thin %to my knowledge
			%		 	\end{itemize}
		
	\end{itemize}
\end{frame}



\begin{frame}[label = motivation]
	\frametitle{Motivation and Research Question} 
	\begin{itemize}

			\item Workers graduating into a recession would likely match to a lower level starting jobs than their luckier counterparts (Devereux 2002)
			%\item economists got good jobs during the recessions of 1980s, 1990s, 2000s than in surrounding years (Oyer 2006) 
			%\vspace{2mm}
		\begin{itemize}
			\item first job placement is important in explaining the long-term losses (Kwon et al 2010, Oreopoulos et al. 2012)
			\vspace{1 mm}
			\item how long the effects remain depends on the availability of switching (Van den Berge 2018, Cockx and Ghirelli 2016)
		\end{itemize}
		

%				\vspace{2mm}
%		\item How does the entry conditions affect economists' career?
%		\begin{itemize}
%			\item short run: initial placements
%			\item long run: occupational switching and publications 
%		\end{itemize}
		\vspace{1.5 mm}
	\item Set up the theoretical model to explain what drives the persistent outcomes for economics PhD
		\vspace{1.5 mm}
	\item Test the model's predictions using detailed information on career paths and productivity measures  available on the web
			\begin{itemize}
		\item short run: initial placements
			\vspace{1 mm}
		\item long run: occupational choices and publications 
	\end{itemize}
%		\begin{itemize}
%			\item demand for research is pro cyclical
%			\item switching careers early is costly, especially for academic economists %\hyperlink{tenure}{\beamerbutton{pic}}
%			%\vspace{1.5mm}
%					\vspace{0.mm}
%			\begin{itemize}
%			\item how do economists develop human capital?
%			
%			\end{itemize}
%			\item evaluate whether the effects would differ by the fields  like college majors (Altonji et al. 2014)
%		\end{itemize}
	
%	\item {\color{red}{task-specific / signaling story}}
	\end{itemize}
\end{frame}


%%%%%%%%%%%%%%%%%%%%%%%%%%%%%%%%%%% LATER %%%%%%%%%%%%%%%%%%%%%%%%%%%%%%%%%%%%%
%\begin{frame}
%	\frametitle{Mechanism: \color{red}{Think again..or remove}} 
%	\begin{itemize}
%		\item Evaluate job mobility to assess the impact of entry economic conditions
%		\begin{itemize}
%			\item recessionary cohorts would take longer periods of time to find a job match
%			\item spending time in bad matches would lead to wage losses and would result in wrong investment in human capital
%			\item  the disparities in human capital are the important channel through which the effects of graduating in a bad economy will remain
%		\end{itemize}
%		\item Not clear what types of human capital economists would develop
%		\item Develop a theoretical model of human capital development to understand the job mobility of economists 
%		\begin{itemize}
%			\item provides the testable hypothesis on whether the entry condition effects remain in the long-run
%		\end{itemize}
%	\end{itemize}
%\end{frame}
%%%%%%%%%%%%%%%%%%%%%%%%%%%%%%%%%%% LATER %%%%%%%%%%%%%%%%%%%%%%%%%%%%%%%%%%%%%



\begin{frame}
	\frametitle{Preview: Findings}
	\begin{itemize}
			\item Demand for economists is procyclical
			\begin{itemize}
				\item fluctuations are primarily driven by the academic tenure-track positions in the US
			\end{itemize}
			\vspace{1mm}
			\item Entry conditions would affect the initial placement outcomes
			\begin{itemize}
				\item  recessionary cohorts are less likely placed in tenure-track academic positions in US
				\item quality of placement would not be affected
			\end{itemize}
			\vspace{1mm}
			\item Recessionary cohorts are less likely to work in academia in {\small  long run}
			\begin{itemize}
				\item the cohorts publish fewer journal articles in top 50 journals
			\end{itemize}
			\vspace{1mm}
			\item Economists rarely switch the occupations over time even if the entry economic conditions were not favorable
			\begin{itemize}
				\item economists develop task-specific human capital
				\item occupations are quite specialized
			\end{itemize}
	\end{itemize}
\end{frame}


\begin{frame}
	\frametitle{Road Map}
\begin{enumerate}
	\item Literature Reviews
	\vspace{2.6 mm}
	\item Data
	\vspace{2.6 mm}
	\item Theoretical Model
	\vspace{2.6 mm}  
	\item Empirical Results
	\vspace{2.6 mm}  
	\item Conclusion
\end{enumerate}

\end{frame}




\begin{frame}
	\frametitle{Contribution: Persistent effects of Entry condition} 
	\begin{itemize}
		\item Analyze the effect of entry conditions onto the labor market outcomes overtime (Kahn 2010, Oreopoulos et al. 2012, Schwandt and von Wachter 2019, Yu et al. 2014,  Maclean, 2015, Ball 2021)
		\begin{itemize}
			\item earnings, labor supply, health, family formation, crimes
			\item effects vary by education, major, race, institutional settings (Altonji et al. 2014, Beiler 2017, Choi et al. 2020, Liu et al.)
		\end{itemize}
		\vspace{1 mm}
		\item Literature estimates the effects on the outcome using cell-level model and the coefficients on entry conditions allow to vary with experience
		\begin{itemize}
			\item Mincerian proxy for the labor market entry and rarely takes account occupational choice
		\end{itemize}
		\item I use an individual-level model and do not allow it to vary by experience
		\begin{itemize}
			\item individual characteristics are observable and important sources of variations
			\item shocks mainly affect the initial placements and economists are immobile after 
			\item my  data includes exact timing of entry and almost complete employment histories
		\end{itemize}
     %	\item Less freedom of entering the market
		
	%	\item write Oyer's findings, and my findings why different
	\end{itemize}
\end{frame}


\begin{frame}
	\frametitle{Contribution: Occupation Choice} 
	\begin{itemize}
		\item Job mobility plays a crucial role in recovering from the early damages  (Van den Berge 2018, Cockx and Ghirelli 2016) 
		\vspace{1.6 mm}
		%\item Effects vary by education levels, race, institutional settings 
		%\item Unlucky college graduates tend to work in less attractive occupations / firms
%		\begin{itemize}
%			\item start and stay longer in lower-wage occupations (Altonji et al. 2016) and industries (Oreopoulos et al.
%			2012)
%			\item higher-earning majors typically fare substantially
%			better in recessions relative to lower earning majors
%		\end{itemize}
		%\item I find 
		\item Human capital formation vs Signaling 
			\begin{itemize}
				\item initial investment in skills specific to occupation keeps a person on a certain career trajectory (Gibbons and Waldman 2004, 2006)
				\item bad signaling from starting in a less favorable job hinders unlucky graduates to from switching occupation when recovers (Nunley et al. 2017)
			\end{itemize}
		\vspace{1.6 mm}
		\item Empirically demonstrates the connection between the task-specific human capital and economist's mobility and how it affects a range of outcomes
		\begin{itemize}
			\item little evidence on signaling
		\end{itemize}		
	\end{itemize}
\end{frame}


%\begin{frame}
%	\frametitle{Contribution} 
%	\begin{itemize}
%		\item Provide a fresh perspective on economist’s human capital formation by analyzing the consequence of the entry economic conditions and job mobility.
%		\begin{itemize}
%			\item analyze the mobility of every individuals
%		\end{itemize}
%		\item To my knowledge, this is the first study that empirically demonstrates the connection between the task-specific human capital and worker’s mobility and how it affects a range of outcomes
%		\item The applications of the model and its predictions are open to the markets in which labor is the most essential input
%		\item General guidelines for the current cohorts who will graduate during pandemic
%	\end{itemize}
%\end{frame}





\begin{frame}[label=Data]
	\frametitle{Data}
	\begin{itemize}
		\item Collect the following data sets to trace economists' career
		\begin{itemize}
			\item list of job postings from JOE
			\begin{itemize}
				\item hiring institution, position, JEL classifications, job descriptions  
			\end{itemize}
		\vspace{1 mm}
			\item ProQuest Dissertations \& Theses Global
			\begin{itemize}
				\item collect the doctoral dissertations by institutions, year of publications, economics (related) classification, subject codes
				\item $\sim$ 4,000 graduates from top 32 programs in U.S. between 2004--2012 
			\end{itemize}
		\vspace{1 mm}				
			\item Scrape CVs on the web or Linkedin experience profile
			\begin{itemize}
				\item collect employment history until 2020
				\item gender and post secondary education information
			\end{itemize}
		\vspace{1 mm}
			\item Publication information from EconLit
%			\item econphd.net rankings 2004
%			\item 2004 National Study of Post secondary Faculty
%			\begin{itemize}
%				\item time usage survey for faculties in U.S.
%			\end{itemize}
		\end{itemize}
	\vspace{1 mm}
		\item Construct the matching algorithm to compile all data  \hyperlink{appendix}{\beamerbutton{appendix}}
	\end{itemize}
\end{frame}


\begin{frame}
	\frametitle{Cyclical Demand for Economics PhD}

\begin{figure}
	\centering
%	\caption{\small Number of JOE Listings and U.S. institutions}
\includegraphics[width=0.45\textwidth]{"/Users/yeabinmoon/Dropbox (UH-ECON)/JMP/writing/figures/temp/figure1".png} 
\includegraphics[width=0.45\textwidth]{"/Users/yeabinmoon/Dropbox (UH-ECON)/JMP/writing/figures/temp/figure2_update".png} 
%	\captionsetup{width=1.0\textwidth}
%	\caption*{\footnotesize For the left panel, I count the number of job postings in JOE Listings by sections. The numbers are totalled based on academic year not calender year. For the right panel, I summarzie the count in the Carnegie Classification of Institutions of Higher Education in 2010.}
\end{figure}
			
%			\item Supply side story
%			\begin{itemize}
%				\item entering into grad school is not affected much by the economy
%				\item selection process is very selective. quite constant supply
%			\end{itemize}
\begin{itemize}	
\item Total postings decreased by 22 percent between 2008--2010
\begin{itemize}
	\item Largest drop occurred for the U.S. academic postings (about 45 \%)
\end{itemize}
%\item Full-time US academic postings showed the largest decline (45 percent)  
\end{itemize}
\end{frame}



\begin{frame}
	\frametitle{Descriptive Statistics}
	\input{"tables/table1"}
\end{frame}


\begin{frame}
	\frametitle{Theoretical Framework}
	\begin{itemize}
		%\item Different theories would lead to different expectations about the persistent impact of entry conditions
		%\item The effect of the initial placement is particularly important for economists
		\item Human capital accumulation is largely determined during the first decade of one's career in high skill occupations (Rosen 1990, O'Flaherty and Siow 1995)
		\begin{itemize}
			\item problem would be critical at research universities, in which tenure decisions are determined within 5-7 years
		\end{itemize}
		\vspace{1.5 mm}
		\item Job mobility would raise  questions on the transferability of skills 
		\begin{itemize}
			\item more costly for  whose skills are not transferable across jobs
		\end{itemize}
			\vspace{1.5 mm}
		\item If task-specific human capital is an integral part of the skill-acquisition process, then cohort effect could arise (Gibbons and Waldman 2006, Jin and Waldman 2019)
	\end{itemize}
\end{frame}


\begin{frame}
	\frametitle{Task-specific Human capital}
	\begin{itemize}
		\item Concept of measuring the transferability of labor market skills
		\begin{itemize}
			\item similar to occupation(or firm) specific human capital
			\item value of human capital depends on the tasks not the workplace
			\begin{itemize}
				\item valued similarly in occupations where similar tasks are performed
			\end{itemize}
%		\item suppose there are two types of tasks: research and teaching
%		\item both types are in general productive in many occupations, and occupations combine the two tasks in different ways.
%		\item called specific because they are only productive in occupations where similar tasks are performed
%		\item different from general skills or occupation specific skills
	%	\item appreciated in occupations where similar tasks are performed
		
		\end{itemize}
	%	\item Need a panel of complete job histories with information on tasks performed in occupations
	%\vspace{1 mm}
		\vspace{1 mm}
	\item Literature use occupational and industry codes from the census
	\begin{itemize}
		\item change in occupation means the skills required for new occupations would be substantially different from those used in the old 
		\item need to build another index because of the small range of occupations economists would work at
	\end{itemize}
		\vspace{1 mm}
	\item I define occupations into the following categories
	\begin{itemize}
		\item R1 university
		\item All other universities in US
		\item Research organization or governmental agencies in US
		\item Foreign institute
		\item Private sector
	\end{itemize}
	\end{itemize}
\end{frame}

\begin{frame}[label=Occupation]
	\frametitle{Definition of Occupations}
	\begin{itemize}
	
			\item Faculties in R1 university spend less time teaching compared to all other universities in US%  \hyperlink{teaching}{\beamerbutton{appendix}}
			\input{"tables/table2ii"}
			\item Research organization in the U.S. does not require teaching, and the research goal would not be the same as the universities
			\vspace{1.5 mm}
			\item Foreign institutes would be different from the U.S. counterparts
			\begin{itemize}
				\item most international universities have different promotion policies from  US (Smeets et al. 2006)
			\end{itemize}
		\vspace{1.5 mm}
			\item Using natural language process on the job descriptions, find a few words in private sectors mostly  \hyperlink{NLP}{\beamerbutton{appendix}}
%				\begin{itemize}
%					\item possibly different skills requirement
%				\end{itemize}
		\end{itemize}

\end{frame}





\begin{frame}
	\frametitle{Model}
	\begin{itemize}
		\item Based on the from the Gibbons and Waldman (2004 and 2006)
		\item Define occupation $o$ as the collection of firms having the same task
		%\item Purpose of modeling is to show whether workers human capital is valued more depends on the initial placements; and that workers more likely move where the tasks are similar
%		\begin{itemize}
%			\item switching firms or occupations for individuals is defined as having different firms or occupations in $t$ compared to  $ t- 1$.
%			\item all firms are contained within occupations, so individuals can only switch occupations if they also switch a firm
%		\end{itemize}
		\item A firm $f$ assigns the combinations of tasks $\left\lbrace 1,...,J\right\rbrace $ to a worker.
		
%		\item Suppose that task-specific output in a firm $f$ within $o$ is produced by combining multiple tasks, denoted by $j=1,...,J$
%		\begin{itemize}
%			\item Occupations combine the tasks in different ways, and let $\beta_o^j \in [0,1]$ be the relative weight on the  task $j$
%		\end{itemize}
%		
%		\item Worker $i$'s task-specific output $Y$ working at $f$ in $o$ and $t$ 
		\item $i$ produces cumulative task-task-specific output $Y_{ifot}^j$
		\begin{equation}
			\log{Y_{ifot}^j} = \sum_{j} \beta_{o}^{j} a_{iot}^j + \mu_{if} \;\;\text{where } \sum_{j} \beta_{o}^j = 1 \;\;\text{for all }o=1,...,O
		\end{equation}
		\begin{itemize}
			\item $\beta_{o}^{j} $ is the share of time a worker spends on average in the task $j$ in  $o$
			\vspace{0.8 mm}
			\item $a_{iot}^j$:  $i$'s productivity for task $j$ at  $o$ and time in labor market $t$
			\vspace{0.8 mm}
			\item $\mu_{if}$ denotes the match quality between $i$ and $f$ 
		\end{itemize}
	\end{itemize}
\end{frame}


\begin{frame}
	\frametitle{Model - continue}
	\begin{itemize}
		\item Productivity depends on initial endowment and experience  
%		$a_{iot}^j$ is determined by a person's initial endowment in each task at entry ($\alpha_{i}^j$)  and the human capital accumulated in the labor market
		\begin{equation}
			a_{iot}^j = \alpha_{i}^j + \gamma_o H^j_{it}
		\end{equation}
		where
		\begin{itemize}
			\item $\alpha_{i}^j$: initial endowment for the task $j$
			\item $\gamma_o$: return to human capital on occupation $o$ 
			\item$H_{it}^j$ is the human capital accumulated in task $j$ until time period $t$
			\begin{equation}
				H_{it}^j = \lambda_{o'}^j \text{Exp}_{io't}
			\end{equation}
			$\text{Exp}_{io't}$ denotes the previous tenure in occupation $o'$ %(to simplify exposition)
		\end{itemize}
		%\;\;	\;\;\;\;\;$\gamma_o$ is the return to human capital on occupation $o$ 
		%	\item $H_{it}^j$ is the human capital accumulated in task $j$ until time period $t$
		

		\item Hence,
	\end{itemize}
	\begin{align}
		\log{Y_{ifot}^j} = \gamma_o&\left[ \sum_j \beta_{o}^j \left(\lambda_{o'}^j \text{Exp}_{io't} \right) \right] + \sum_j \beta_{o}^j \alpha_{i}^j + \mu_{if}\\
		&\text{where} \sum_{j} \beta_{o}^j = 1 \;\;\text{for all }o=1,...,O \notag
	\end{align}
\end{frame}


\begin{frame}
	\frametitle{Model - continue}
\begin{align}
	\label{eq:eq9}
	\log{Y_{ifot}^j} = \gamma_o&\overbrace{\left[  \sum_j \beta_{o}^j \left( \underbrace{H_{it}^{j}}_{\lambda_{o'}^j \text{Exp}_{io't}}\right)  \right]}^{\text{Task}_{iot}} + \underbrace{\overbrace{\sum_j \beta_{o}^j \alpha_{i}^j }^{\text{m}_{io}}+\mu_{if}}_{\text{Match quality}}\\
	&\text{where} \sum_{j} \beta_{o}^j = 1 \;\;\text{for all }o=1,...,O \notag
\end{align}

	
	\begin{itemize}
		\item $\text{Task}_{iot}$ is a measure of task-specific human capital valued by $o$
		\item $\text{m}_{io}$ is the match quality between $i$ and occupation $o$
		\item assume $\mu_{if}$ is random and does not develop over time
		%\item To simplify exposition, consider there are three tasks, for example research, teaching, and others
		\item At entry $H_{it}^j=0$, so  initial output is determined by match qualities
%		\begin{itemize}
%			\item Initial placement effects are reflected through the match quality
%			\item Assume the match quality with a firm is conditionally random 
%			\item Match quality with occupation would be affected by economic condition
%		\end{itemize}
	\end{itemize}
	
\end{frame}


\begin{frame}
	\frametitle{Incorporating Entry Condition of Business Cycle}
	\begin{itemize}
		\item Impose two more assumptions to reflect the effect of economic conditions at entry
		\begin{block}<1->{Assumption 1. most workers are research-oriented}
			$\alpha_{i}\equiv \left( \alpha_{i}^1,...,\alpha_{i}^J\right) \equiv m(X_i)+e_{it}, \text{ where }  \alpha_{i}^1 \geqq \max_{j\ne 1} \alpha_{i}^j $  
		\end{block}
	\begin{itemize}
		\item $j=1$ indicates economics-research task 
	\end{itemize}
\vspace{1 mm}
%		\begin{block}<1->{Assumption 2. Finding an reserch-heavy occupation is procyclical}
%			Demand from research university is procyclical
%		\end{block}	
%		\item The two assumptions yields the following theorem
        \item Let $\bar{t}$ and $u_{\bar{t}}$  denote the graduation year of $i$ and economic condition at the moment 
		\begin{block}<1->{Theorem 1. mismatch arises during the bad times at the entry}
		$\text{If } u_{\bar{t}}< u'_{\bar{t}},\text{ then }\mathop{{}\mathbb{E}}_{i}\left[ m_{io} \mid u_{\bar{t}}, \sum_{j} H_{it}^{j} = 0 \right] > \mathop{{}\mathbb{E}}_{i}\left[ m_{io} \mid u'_{\bar{t}}, \sum_{j} H_{it}^{j} = 0 \right]$
			%$\mathop{{}\mathbb{E}}_{i} \left[ m_{io}\right | u_{t} < u_{t'} \text{ and } \sum_j H_{it}^{j}=0 \right] $
			%> \mathop{{}\mathbb{E}}_{i} \left[ m_{io}\right | u_{t'} > u_{t} \text{ and }H_{it'}^j=0\; \forall j]$ 
		\end{block}
		\begin{itemize}
			\item consistent with Bowlus (1995)
		\end{itemize}
%		\item Now consider the economist' labor market characteristics and production dynamics
	\end{itemize}
\end{frame}

\begin{frame}
	\frametitle{Mobility}
	\begin{itemize}
		\item If $i$ does not switch the occupation, the following corollary is derived:
%		\begin{block}<1->{Assumption 3. Job switching is prohibited over the few years}
%			Economists work under up-or-front policy	
%		\end{block}
		\begin{block}<1->{Corollary 1.}
			$\text{If }u_{\bar{t}} < u'_{\bar{t}} \text{ and } i \text{ did not switch }o,$\\
			$\text{ then } \mathop{{}\mathbb{E}}_{i} \left[Y_{ifot}^1 \mid u_{\bar{t}}, X_i \right] > \mathop{{}\mathbb{E}}_{i} \left[Y_{ifot}^1 \mid u'_{\bar{t}}, X_i \right]\;\;\text{for all}\; t$
		\end{block}	
	\vspace{1 mm}
		\item The gap in productivity is driven by the two channels
		\begin{itemize}
			\item unfavorable economic conditions result in mismatch 
			\item unfavorable human capitals are developed according to the tasks 
		\end{itemize}
	\vspace{1 mm}
		\item Consider how the task-specific human capital would be valued if a worker would switch occupations
		\begin{itemize}
			\item To make an exposition simpler, examine two-task model $J = \left\lbrace R, T \right\rbrace $
		\end{itemize}
	\end{itemize}
\end{frame}



%
%\begin{frame}
%	\frametitle{Model with 3 Tasks}
%	\begin{itemize}
%		\item Follow the previous notations
%		\begin{align*}
%			\log{Y_{ifot}} &= \gamma_o  \overbrace{ \left\lbrace \beta_o^1 H_{it}^1+\beta_o^2 H_{it}^2+\beta_o^3 H_{it}^3\right\rbrace}^{\text{Task Tenure}}+\overbrace{\underbrace{\beta_{o}^{1} \alpha_{i}^1 +\beta_{o}^{2} \alpha_{i}^2+\beta_{o}^{3} \alpha_{i}^3}_{\text{with o}}+\underbrace{\mu_{if}}_{\text{with f}}
%			}^{\text{match quality}}\\
%			&=\gamma_0 T_{iot}+m_{io}+\mu_{if}
%		\end{align*}
%		where $\sum_{j} \beta_{o}^j=1$, and $j=1,2,3$ indicate research, teaching, and others, respectively
%		\item When entering the market, there is no human capital accumulated
%		\begin{itemize}
%			\item Initial placement effects are reflected through the match quality
%			\item Assume the match quality with a firm is conditionally random 
%			
%		\end{itemize}
%		\item Match quality with occupation would be affected by economic condition
%	\end{itemize}
%\end{frame}

%
%\begin{frame}
%	\frametitle{Model with 2 tasks: Task Tenure with Mobility}
%	\begin{itemize}
%		\item A worker was employed for 1 year at occupation $o'$ that specializes research ($\lambda_{o'}=.6$). Then $H^R=0.6$ and $H^T=0.4$
%		\begin{itemize}
%		\item if she stays at the same occupation ($\beta_{o} = .6$)
%		\begin{itemize}
%			\item $0.6\times 0.6 + 0.4 \times 0.4 = 0.52$
%		\end{itemize}
%		\item if she moves to more research occupation ($\beta_{o} = .8$)
%		\begin{itemize}
%		\item 		$0.8\times 0.6 + 0.2 \times 0.4=0.56$
%		\item worker has accumulated more human capital in research than teaching and that the target occupation rewards more than the source occupation
%		\end{itemize}
%		\item if she moves to less research occupation ($\beta_{o} = .4$)
%	\begin{itemize}
%		\item 		$0.4\times 0.6 + 0.6 \times 0.4=0.48$
%	\end{itemize}
%\end{itemize}
%	\item Similar results are derived for a worker was employed for 1 year an occupation $o'$ that specializes teaching
%	\item A worker was employed for 1 year an occupation $o'$ in which specializes the skills evenly ($\lambda_{o'}=.5$)
%\begin{itemize}
%	\item worker's tenure does not change regardless of moving
%	\end{itemize}
%\end{itemize}

%					\end{itemize}
%		\item demonstrate that how much the task tenure is valued depends on the degree of specialization in the previous occupations
%\item Therefore,  transfer-ability of task-specific human capital depends on the degree of specialization in the previous institutions
%			\item Workers accumulate more portable human capital in specialized occupations where $\beta_{o'}$ is close to one or zero than it close to 0.5
%			\item Given that tasks are highly polarized for academic researchers, disregard the case for 0.5 


%\end{frame}




\begin{frame}
	\frametitle{Task Tenure with Occupational Choice}
	\begin{itemize}
		\item $o'$ and $o$ indicate the source and target occupations, respectively
	\begin{block}<1->{Proposition}
\begin{align*}
	&\text{For }\lambda_{o'}^R>0.5,\text{ task-tenure is valued more if moves to }\beta_o^R > \lambda_{o'}^R\\
	&\text{For }\lambda_{o'}^R<0.5,\text{ task-tenure is valued more if moves to }\beta_o^R < \lambda_{o'}^R\\
	&\text{For }\lambda_{o'}^R = 0.5,\text{ task-tenure does not change regardless of moving}
\end{align*}
	\end{block}
%	\begin{itemize}
		%\item proof
		\item How the task tenure is valued depends on the degree of specialization in the source occupation
		\begin{itemize}
			\item one's tenure is valued more if the target occupation more specializes (close to 1) than the source occupation
			\item If the source occupation is very general (close to $0.5$), switching does not have any merits
		\end{itemize}
		\item Now consider the implication for job mobility
	\end{itemize}
\end{frame}




\begin{frame}
	\frametitle{Occupational Choice}
	%	\[
	%	log{w_{iot}}=\kappa_o + \gamma_o\text{Task}_{iot}+\text{mc}_{io}
	%	\]
	\begin{itemize}
		%\item Wages is determined by worker's productivity multiplied with the occupation specific skill price $K_o$
		%\item Workers search over to maximize output%, and hence moving decision depends on the transferability of task-specific human capital $\omega_{ist}$, the task match $u_{is}$ and the occupation-specific return to human capital $\gamma_s$
		%\end{itemize}
		\item Suppose research oriented worker $i$ started working at $f'$ within teaching-heavy $o'$ have an option to switch
		\begin{itemize}
			\item switching entails the switching cost $x_{o't}$ 
		\end{itemize}
		\item $i$ faces 
		\begin{equation}
			\max_{o',o} \left[Y_{if'o't}, Y_{ifot}-x_{o't} \right] 
		\end{equation}
		\item Improvement on match-up qualities and returns to task tenure would make a shift more likely, but there is a loss from the task tenure according to the proposition when move
	\end{itemize}
	%	\begin{align}
	%		\overbrace{\left(m_{io}-m_{io'} \right)}^{\text{Counterfactual of initial condition}} + \left( \mu_{if}-\mu_{if'}\right) +&\left( \gamma_o-\gamma_{o'}\right) \text{Task}_{io't} \\
	%		&> \gamma_o \underbrace{\left(\text{Task}_{io't} - \text{Task}_{iot} \right)}_{\text{potential loss}} \notag
	%	\end{align}
\begin{align}	\label{eq:eq10}
	\left(m_{io}-m_{io'} \right) + &\left( \mu_{if}-\mu_{if'}\right)+\left( \gamma_o-\gamma_{o'}\right) \text{Task}_{io't} \\
	&>\gamma_o\underbrace{\left[ \left( \beta_{o'} - \beta_{o}\right) \left( H_{it}^R-H_{it}^T\right) \right]}_{\text{potential loss}} +  \underbrace{x_{o't}}_{\text{switching cost}} \notag
\end{align}
	
	
	%	\begin{itemize}
	%		\item Improvement on match-up qualities or or returns to task tenure would make worker switch to produce more
	%		\item However if the potential loss of task tenure were large, staying at the same occupation would generate more output    
	%	\end{itemize}
\end{frame}


\begin{frame}
	\frametitle{Empirical prediction}
	
\begin{align}	\label{eq:eq10}
	\left(m_{io}-m_{io'} \right) + &\left( \mu_{if}-\mu_{if'}\right)+\left( \gamma_o-\gamma_{o'}\right) \text{Task}_{io't} \\
	&>\gamma_o\underbrace{\left[ \left( \beta_{o'} - \beta_{o}\right) \left( H_{it}^R-H_{it}^T\right) \right]}_{\text{potential loss}} +  \underbrace{x_{o't}}_{\text{switching cost}} \notag
\end{align}
	
	%	\begin{align}
	%	\overbrace{\left(m_{io}-m_{io'} \right)}^{\text{Counterfactual of initial condition}} + &\left( \mu_{if}-\mu_{if'}\right)+\left( \gamma_o-\gamma_{o'}\right) \text{Task}_{io't} \\
	%	&>\gamma_o\underbrace{\left[ \left( \beta_{o'} - \beta_{o}\right) \left( H_{it}^R-H_{it}^T\right) \right]}_{\text{potential loss}}  \notag
	%\end{align}
	\begin{itemize}
		\item Potential loss is governed by two factors
		\vspace{1 mm}
		\begin{itemize}
			\item how similar the tasks between occupation $o$ and $o'$, $|\beta_o-\beta_{o'}|$
			\begin{itemize}
				\item if the source occupation is very general, there would be no loss
			\end{itemize}
		\vspace{1 mm}
			\item how much human capital accumulated from the previous occupations
		\end{itemize}
		%		\item Assume that the workers stay the previous institutions for a while  
	
	\end{itemize}
\end{frame}





\begin{frame}
	\frametitle{Discussion: Overview of the model's contributions}
	\begin{itemize}
		
		\item  If economists’ human capital is not task-specific, the markets would be similar to the high skilled industry
		\begin{itemize}
			\vspace{0.6 mm}
			\item  the workers would solve the mismatch by switching, and hence the effect of entry conditions would be away soon
		\end{itemize}
		\vspace{1 mm}
		
			\item If workers' human capital is task specific, there are two more cases
		\begin{itemize}
			\item the economist’s tasks are specialized (distances  are significant)
			\begin{itemize}
				\item they would less likely switch because they might risk losing the human capital 
				\item the initial effects would remain 
			\end{itemize}
		\vspace{1 mm}
			\item the economist’s tasks are general (distances are small)
			\begin{itemize}
				\item  economists would more easily switch the occupation, and hence the initial placement effects are less likely permanent
			\end{itemize}
		\end{itemize}

	\end{itemize}
\end{frame}





\begin{frame}
	\frametitle{Prediction I: Initial Placements}
	\begin{itemize}
		\item I first test whether the entry economic conditions predict the initial placement outcomes: 
		\begin{itemize}
			\item for individual $i$, cohort $c$, department $d$, fields of study $f$
		\begin{equation}
			\label{eq:eq1}
			y_{icdf} = \beta \text{ec}_{c}+\gamma X_{i}+\lambda_{d}+\theta_f  +\epsilon_{icdf}
		\end{equation}
	where $ec_c$ indicates the economic conditions at graduation for $c$
%	\item $y_{icdf}$  is whether $i$ held a tenure track position in R1 university
	\item  approximate $ec_c$  using the unemployment rate as of October at the one year before graduation
	\item $X_i$ includes an indicator for receiving bachelor degrees in the U.S. and gender
	\item  $\beta$ would be unbiased as long as the average quality of economists entering the market is not systematically related to $ec_c$
\end{itemize}
	\end{itemize}
\end{frame}




{
	\setbeamerfont{frametitle}{size=\small}
	\begin{frame}
		\frametitle{Effect of entry conditions on the initial placement in R1 universities}
		\input{"tables/table3"}
	\end{frame}
}

{
	\setbeamerfont{frametitle}{size=\small}
	\begin{frame}
		\frametitle{Effect of entry conditions on the initial placement in rankings}
		\input{"tables/table4"}
	\end{frame}
}

{
	\setbeamerfont{frametitle}{size=\small}
	\begin{frame}
		\frametitle{Effect of entry conditions on the initial placement: multinomial logit}
		\input{"tables/table5"}
	\end{frame}
}

\begin{frame}
	\frametitle{Cohort Effects at Entry}
	\begin{figure}
		\centering
		%	\caption{\small Number of JOE Listings and U.S. institutions}
		\includegraphics[width=0.7\textwidth]{"/Users/yeabinmoon/Dropbox (UH-ECON)/JMP/presentations/oct/figures/fig5".png} 
%		\includegraphics[width=0.45\textwidth]{"/Users/yeabinmoon/Dropbox (UH-ECON)/JMP/presentations/oct/figures/fig2".png} 
		%	\captionsetup{width=1.0\textwidth}
		%	\caption*{\footnotesize For the left panel, I count the number of job postings in JOE Listings by sections. The numbers are totalled based on academic year not calender year. For the right panel, I summarzie the count in the Carnegie Classification of Institutions of Higher Education in 2010.}
	\end{figure}
	
	%			\item Supply side story
	%			\begin{itemize}
		%				\item entering into grad school is not affected much by the economy
		%				\item selection process is very selective. quite constant supply
		%			\end{itemize}
\end{frame}



\begin{frame}
	\frametitle{Prediction 2: Long-run Placements}
	\begin{itemize}
		\item I now test whether the entry economic conditions predict the occupational choice in the long run
		\begin{itemize}
			\item Using the same specification (\ref{eq:eq1}), the dependent variable is whether one work at R1 university nine years after graduation
		\end{itemize}
	\vspace{2 mm}
		\item The model predicts that the effect will remain if economists develop task-specific human capitals 
		%\item Find that unemployment has a significant impact on the initial placements as expected
	\vspace{2 mm}
		\item Also, if one had a higher switching cost, the effects would be stronger
%		\begin{itemize}
%			\item also, if one had a higher switching cost, the effects would be stronger
%		\end{itemize}
	\end{itemize}
\end{frame}

{
	\setbeamerfont{frametitle}{size=\small}
	\begin{frame}
		\frametitle{Effect of entry conditions on the placement in R1 universities 4 years after}
		\input{"tables/tableplus"}
	\end{frame}
}

{
	\setbeamerfont{frametitle}{size=\small}
	\begin{frame}
		\frametitle{Effect of entry conditions on the placement in R1 universities 9 years after}
		\input{"tables/table6"}
	\end{frame}
}


\begin{frame}[label = discussion]
	\frametitle{Discussion}
	\begin{itemize}
%		\item As the model predicts, the entry economic conditions remain in the long-run
		\item Note that the magnitudes of the effects is way smaller than the initial impact
		\begin{itemize}
			\item some individuals might switch the occupations but not enough to close the initial gaps
			%\item if so, possible to predict who did switch? 
		\end{itemize}
		\item Further test whether one ever switch occupation or firm   \hyperlink{mobility}{\beamerbutton{appendix}}
		\begin{itemize}
			\item as the model predicted:
			\begin{itemize}
				\item less likely switch the occupation
				\item if one switched, it would happen within the same occupations at early periods
			\end{itemize}
		\end{itemize} 
			\item Other explanations: entry conditions would serve as a signal of ability
			\begin{itemize}
				\item its importance as a signal declines over time as more information of true ability is revealed \hyperlink{signal}{\beamerbutton{appendix}}
			\end{itemize} 
		%its importance as a signal declines over time as more information of true ability is revealed \hyperlink{signal}{\beamerbutton{appendix}}	\item One might raise the question of whether the entry condition would serve as a signal of ability, and 
	\end{itemize}
\end{frame}




\begin{frame}
	\frametitle{Prediction 3: Productivity}
	\begin{itemize}
		\item Now I test whether the entry economic conditions would affect the economists' productivity
		\begin{itemize}
			\item main measures of research output for academic economists are their publications
			\item for individual $i$, cohort $c$,  department $d$, field of study $f$, year $t$, labor market experience $exp$
				\begin{equation*}
					y_{icdft} = \beta \text{ec}_{c}+\gamma X_{i}+\xi_{d}+\theta_f +\mu_{exp}  +\epsilon_{icdft}
				\end{equation*}
				where $ec_c$ indicates the economic conditions at graduation for $c$
				%\item tenure in the profession, publication outcomes, time to tenure, mobility
				\item $y_{icdft}$ is the number of publications in top 50 economics journals
				\item $X_i$ includes an indicator for receiving bachelor degrees in the U.S. and gender 
			\end{itemize}
			%\item $exp$ denotes years of potential labor market experience
			%		\item $\text{exp}_{it}$ denotes the potential labor market experience
			%		\begin{itemize}
				%			\item only makes sense when outcome is not specific to certain occupations like wage
				%			\item if $Y$ is num publications, it should count  academic experience only (updating)
				%		\end{itemize}	
		%		\item Identifying assumption in Model I would be weak
		%		\begin{itemize}
			%			\item those who end up in better jobs later may
			%		\end{itemize}
	\end{itemize}
\end{frame}

{
	\setbeamerfont{frametitle}{size=\small}
	\begin{frame}
		\frametitle{Effect of entry conditions on the Publications}
		\input{"tables/table7"}
	\end{frame}
}

\begin{frame}
	\frametitle{Cohort Effects over time}
	\begin{figure}
		\centering
		%	\caption{\small Number of JOE Listings and U.S. institutions}
		\includegraphics[width=0.48\textwidth]{"/Users/yeabinmoon/Dropbox (UH-ECON)/JMP/presentations/oct/figures/figa".png} 
		\includegraphics[width=0.48\textwidth]{"/Users/yeabinmoon/Dropbox (UH-ECON)/JMP/presentations/oct/figures/figb".png} 
		%	\captionsetup{width=1.0\textwidth}
		%	\caption*{\footnotesize For the left panel, I count the number of job postings in JOE Listings by sections. The numbers are totalled based on academic year not calender year. For the right panel, I summarzie the count in the Carnegie Classification of Institutions of Higher Education in 2010.}
	\end{figure}
	
	%			\item Supply side story
	%			\begin{itemize}
		%				\item entering into grad school is not affected much by the economy
		%				\item selection process is very selective. quite constant supply
		%			\end{itemize}
\end{frame}

\begin{frame}[label = robustness]
	\frametitle{Robustness Check}
	\begin{itemize}
	\item In the analysis above,  assume that the macroeconomic conditions at graduation represent an exogenous labor demand shock
	\begin{itemize}
		\item  the average quality of graduates who enters the market is not systematically associated with the economic conditions
		\end{itemize}
	\vspace{1 mm}
	\item Note that five years of study is arguably the norm of the economics Ph.D. programs	
%	\item Observe the duration of the study for 60 percent of the sample
	\vspace{1 mm}
	\item Examine the effect of the entry economic conditions on one's decision to delay graduation % \hyperlink{delay}{\beamerbutton{appendix}}
	\begin{itemize}
		\item individuals rank 1 programs would have an option to delay 
		\item revisit the previous findings using individuals from rank2 and rank 3 programs % \hyperlink{rank2}{\beamerbutton{appendix}}
	\end{itemize}
\end{itemize}
\end{frame}


{
	\setbeamerfont{frametitle}{size=\small}
	\begin{frame}[label = delay]
		\frametitle{Robustness Check: Effect of of economic conditions on delaying graduation}
		\input{"tables/table9"} 
%		\hyperlink{robustness}{\beamerbutton{slide}}
	\end{frame}
}

{
	\setbeamerfont{frametitle}{size=\small}
	\begin{frame}[label = rank2]
		\frametitle{Robustness Check: Regressions without graduates from rank 1 school}
		\input{"tables/table10"} 
%		\hyperlink{robustness}{\beamerbutton{slide}}
	\end{frame}
}

\begin{frame}
\frametitle{Discussions: Possible Concerns}
	\begin{itemize}
		\item Incomplete data extraction
		\begin{itemize}
			\item about 12 percent of individuals have no records from ProQuest listings 
		\end{itemize}
		\vspace{1 mm}
		\item Possible mismatch between the degree date and market entry
		\vspace{1 mm}
		\item Selection issues on CV / resume
		\begin{itemize}
			\item more successful individuals would complete CVs more often   
			\item intentionally hide the previous positions
		\end{itemize}
		\item Matching errors
		\vspace{1 mm}
		\begin{itemize}
			\item Duplicated names
		\end{itemize}
	\end{itemize}
\end{frame}



\begin{frame}
	\frametitle{Conclusion}
	\begin{itemize}
		\item The entry economic conditions did affect the initial placements and subsequent occupational choice
		\begin{itemize}
			\item recessionary cohorts less likely to work at R1 university and publish less articles
		\end{itemize}
	\item My model points to the direction of mobility and it will make the initial effects remain longer
	\item Although economics profession would reach to almost full employment, the trajectory would depend on the entry economic conditions
	\item I will examine further firm mobility instead of occupation and whether the entry condition would affect the quality of the firms in the long run
	\end{itemize}
\end{frame}



\begin{frame}[label=appendix]
	\frametitle{Fuzzy matching}
	\begin{itemize}
		\item One challenge of the task is scrape text data from the source document and convert them into suitable format
		\begin{itemize}
			\item Scraping - use various APIs 
			\begin{itemize}
				\item might involve legal issues $\rightarrow$ commercial APIs
			\end{itemize}
		\end{itemize} 
		\item Bigger challenge is that there are same institution but were taken as different forms 
		\begin{itemize}
			\item CV, dissertations, rank data, Journal entry
			\item matching economists' names are even more complicated
		\end{itemize}
			\item Employ learning methods from data science literature
			\begin{itemize}
				\item data matching or fuzzy matching (probabilistic data matching)
			\end{itemize}
	\end{itemize}
\end{frame}


\begin{frame}
	\frametitle{Steps}
	\begin{itemize}
	\item N-grams: a set of co-occurring words within a given sentence (Wang et al. 2006)
	\begin{itemize}
		\item collect the words in the sentence having more meaning
	\end{itemize}
	\item TF-IDF: count the word occurs in each document
	\begin{itemize}
		\item evaluate how important a word is and (learning)
		\begin{itemize}
			\item very important since the names have only a few words
		\end{itemize}
		\item long computing time ...
	\end{itemize}
	\item Cosine similarity: how close the two sentences is
	\item Matching rates vary
	\begin{itemize}
		\item JOE in US institutions: 89\%
		\item All institutions: 70\%
	\end{itemize}
	\end{itemize}
\hyperlink{Data}{\beamerbutton{back}}
\end{frame}

%
%\begin{frame}[label = teaching]
%	\frametitle{Hours per week teaching credit classes}
%	\input{"tables/table2"}
%	\hyperlink{Occupation}{\beamerbutton{slide}}
%\end{frame}
%

\begin{frame}[label = NLP]
	\frametitle{Job description: Natural Language Processing}
	\begin{itemize}
		\item Analyze the text in the job descriptions from JOE and CSWEP letters (central bank, consulting firms
		\item Most Frequently Appeared Words in job postings
		\begin{itemize}
			\item Tenured track positions: \textbf{research, economics, teaching}, curriculum
			\item Research org: \textbf{research, economics}, teaching
			\item Private: \textbf{research}, economics, communication,work, policy, experience, analysis, skills, quantitative, 
		\end{itemize}
		
		\item Word \textbf{research} and \textbf{teaching} dominates in Academic positions
		\item Diverse range of words are captured in private sector positions
		\begin{itemize}
			\item communication related words are rarely captured in academic positions
		\end{itemize} 
		\item Possibly, different skills are required for the private sectors \hyperlink{Occupation}{\beamerbutton{slide}}
	\end{itemize}
\end{frame}


{
	\setbeamerfont{frametitle}{size=\small}
	\begin{frame}[label = mobility]
		\frametitle{Effect of entry conditions on the Job mobility}
		\input{"tables/table8"}
		\hyperlink{discussion}{\beamerbutton{slide}}
	\end{frame}
}

{
	\setbeamerfont{frametitle}{size=\small}
	\begin{frame}[label = signal]
		\frametitle{Effect of entry conditions on the placement in R1 universities over time}
		\input{"tables/app1"}
		\hyperlink{discussion}{\beamerbutton{slide}}
	\end{frame}
}



\end{document}

